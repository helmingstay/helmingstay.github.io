\documentclass[12pt]{article}
\usepackage[left=0.75in,top=0.6in,right=0.75in,bottom=0.6in]{geometry} % Document margins
%\usepackage{url}
\usepackage{hyperref}
\usepackage[numbers, sort&compress]{natbib}
%\usepackage{citeref}  %Not avail on linux.unm.edu; makes page #s in biblio
%\usepackage{graphicx}
%\usepackage{multicol}

\author{Christian Gunning}
%% a statement of activities contributing to diversity, equity and inclusion in Academia
\title{Activities contributing to\\diversity, equity and inclusion in academia}
\date{}

\begin{document}
\maketitle
\thispagestyle{empty}
\linespread{1.25}\selectfont

My approach to fostering equity and inclusion in academia 
is rooted in the so-called ``imposter syndrome'', wherein 
qualified individuals undervalue their own abilities and 
achievements relative to others.  When surrounded by 
teachers and mentors who look and act like us, it's easy to 
see ourselves {\em becoming} them.  Yet lacking such supportive 
social and environmental queues, we often respond with excessively
harsh critiques of ourselves. Indeed, skilled scientists
{\em excel} at self-criticism, often to their own detriment.
As an educator and mentor, I see my role as building 
core competence, and in helping
bridge gaps between competence and confidence.

I view three elements as crucial to building robust student 
self-confidence: enthusiasm, 
competence, and ownership.  First off, enthusiasm is born from both 
curiousity and familiarity. Less priviledged and less prepared
students are at greater risk from disengagement during dry, abstract
recitations. From the start, I build student engagement 
through exploratory physical activities involving peer interaction.
I draw examples from current events and common 
human experience, and employ multimedia visual aids, including 
XKCD cartoons and segments from professionally produced
video series (.e.g., Crash Course Biology). 
This supports students with a range of learning styles, while
providing engaging material for students to consult outside 
of lecture. Altogether, my goal is to provide accessible and 
memorable classes, which aids can increase class attendence for 
at-risk students.

My second main goal is skill mastery. 
Here I aim to challenge students without intimidating
them. I regularly provide goal-oriented assignments 
that build core quantitative and critical thinking skills 
through a series of manageable steps. I emphasize
fairness and a sense of structure, making expectations 
both reasonable and clear. I also strive to provide
support for less-prepared students, including review
sessions and through institution resources. One 
strategy that I've found particularly 
useful is the informal peer mentoring provided by 
assigning students into small (2-3 person)
study groups. They are {\em requested} to work on
assignments together in these groups, and then provide
individual write-ups. This arrangement provides 
students with a degree of ``social accountability'' where
they can seek informal peer assistance, while still 
providing indidual accountability. In addition, such peer 
mentorship can build mentorship skills in individuals who
might not otherwise have such opportunities.

Finally, I provide frequent opportunities
for student input, including surveys, student-generated data, 
and self-selected final projects.  Active student involvement 
in the learning process provide students with a sense 
of participation and belonging. 

As a mentor, I actively seek out students with strong work 
ethics and a thirst for achievement. I view confidence and 
analytical background as skills that can be gained, whereas

As such, I seek students
who {\em lack} a sense of entitlement. 

As an educator at the University of New Mexico, a 
minority-serving institution in
a historically poor state, I have grappled with the powerful
impacts of socioeconomic inequality on student achievement. 
These deep structural inequalities affect students in more 
and less obvious ways, from financial ability to pursue higher 
education, to self-confidence and access to social capital.
In short, the process of learning can be especially challenging
for working parents, returning veterans, or individuals who
lack community support. In my experience, expressing an 
appreciation of the challenges less privileged students confront
simply to arrive in the classroom can increase students'
comfort levels, freeing them to take ownership of their educations.
I also encourage students to connect with each other,
and to proactively seek support. Finally, I 
regularly solicit (optionally anonymous) feedback 
from students via online surveys, thus provides a 
secure, private channel through which to address discomforts 
and concerns.

Throughout my graduate student career, 
I witnessed too many instances of implicit and explicit 
discrimination, often by well-meaning scientist who were 
unaware of the intense discomfort they created. 
Ultimately, increased academic diversity and inclusion must 
come largely from {\em maintaining} existing diversity, and 
pushing back against a culture of exclusion. At every career 
stage, a significant attrition of diversity has long been 
evident. A culture of inclusion begins first with a sense of 
purpose and belonging. I seek to foster such an inclusive 
environment both by explicitly addressing the myriad unique 
challenges and 
assets that each individual brings to the table, and by
providing a consistent pedagogical framework to foster 
engagement of students from all backgrounds. 

\end{document}


%To this day, mathematics and computer programming are still too often viewed as stereotypically ``male'' pursuits. This is an area where I feel particularly well-suited to foster an environment of inclusion.  

%cognitive diversity

%Explicitly addressing the universality of this experience can help students connect with each other, as well as give them permission to seek support. 
