\documentclass[12pt]{article}
%\usepackage{url}
\usepackage{hyperref}
\usepackage[numbers, sort&compress]{natbib}
%\usepackage{citeref}  %Not avail on linux.unm.edu; makes page #s in biblio
%\usepackage{graphicx}
%\usepackage{multicol}

\author{Christian Gunning}
\title{Activities contributing to diversity, equity and inclusion in academia}
\date{\today}
\begin{document}

%% a statement of activities contributing to diversity, equity and inclusion in Academia

My approach to fostering equity and inclusion in academia 
is rooted in the so-called ''imposter syndrome''.  When surrounded by 
teachers and mentors who look and act like us, it's easy to 
see ourselves {\em becoming} them.  Yet lacking such supportive 
social and environmental queues, we often respond with excessively
harsh critiques of ourselves. Indeed, talented critical thinkers often 
excel at self-criticism to their own detriment.
As an educator and mentor, I see my role as helping
bridge this confidence gap.

In the fight against ``imposter syndrome'', I view three 
elements as crucial: enthusiasm, competence, and ownership.
First, employing hands-on
activities and salient modern-day examples helps build student engagement.
Second, homework: by regularly completing goal-oriented assignments, 
students build a sense of accomplishment, along with the habit of 
breaking large tasks up into small pieces.  Finally, opportunities
for student input in the form of surveys, data generation, and 
self-selected final projects all provide students with a sense 
of participation and belonging. 

As an educator at a minority-serving institution in
a historically poor state, I have also grappled with the powerful
impacts of socioeconomic inequality on student achievement. 
These deep structural inequalities affect students in more 
and less obvious ways, from financial ability to pursue higher 
education, to self-confidence and access to social capital.
In short, the process of learning can be especially challenging
for working parents, returning veterans, or individuals who
lack community support. In my experience, expressing an 
appreciation for the challenges many students must confront
simply to {\em arrive} at each class can increase students'
comfort levels, freeing them to take ownership of their education.
In addition, I encourage students to connect with each other, 
and to proactively seek support.

Ultimately, increasing academic diversity and inclusion must 
come largely from {\em maintaining} diversity, and pushing back
against a culture of {\em exclusion}. At every career stage, a
significant attrition of diversity has long been 
evident. A culture of inclusion begins first with a sense of 
purpose and belonging. I seek to foster such an inclusive 
environment both by explicitly addressing the myriad unique 
challenges and 
assets that each individual brings to the table, as well as 
providing a consistent pedagogical framework to foster 
engagement of students from all backgrounds. 

\end{document}


%To this day, mathematics and computer programming are still too often viewed as stereotypically ``male'' pursuits. This is an area where I feel particularly well-suited to foster an environment of inclusion.  

%cognitive diversity

%Explicitly addressing the universality of this experience can help students connect with each other, as well as give them permission to seek support. 
