\documentclass[12pt]{article}
\usepackage[left=0.75in,top=0.6in,right=0.75in,bottom=0.6in]{geometry} % Document margins
%\usepackage{url}
\usepackage{hyperref}
\usepackage[numbers, sort&compress]{natbib}
%\usepackage{citeref}  %Not avail on linux.unm.edu; makes page #s in biblio
%\usepackage{graphicx}
%\usepackage{multicol}

\author{Christian Gunning}
\title{Teaching Philosophy and Experience}
%\date{\today}
\begin{document}
\maketitle
\thispagestyle{empty}
\linespread{1.25}\selectfont

My teaching philosophy is meaning-centered and skill-oriented.
I teach and mentor students from a wide range of backgrounds, 
many of whom lack strong mathematical or analytical skills. 
In formal teaching and mentorship alike, I prioritize
exploration and student ownership, while facilitating
student development of conceptual and analytical expertise.

My initial focus in course design 
is to find material that students can identify with,
whether through popular culture, humor, or common human 
experience. For example, I often use coin flips to 
illustrate topics in randomness. A simple phenomenon 
that students can both visualize and quickly test, 
coin flips can nonetheless be used to illustrate a range 
of complex phenomena, from Markov chains to the central limit 
theorem. Such material provides a core vocabulary with 
which more complex concepts can be comfortably explored.

My next objective is to reveal previously hidden 
relationship. One example involves 
asking students to imagine searching for their keys 
in one dimension,
along a line or curve, versus two dimensions, a surface or
table-top, versus a cluttered house.  Here we see the {\em 
curse of dimensionality} in action, where a hypothetical 
10-dimensional space can be imagined through it's search 
complexity.  

By posing carefully constructed
questions, I challenge students to identify specific connections,
both through collaborative exploration, as well as trial and error.
For example, students can experimentally explore the relationship
between sample size and estimate variance with dice. What is the 
probability of rolling ``snake eyes''? Let's find out!
This sort of tactile, student-centered exploration provides 
a powerful companion to more analytical material, 
helping increase student motivation and recall.

Finally, I challenge students to pose novel questions based on
their own interests and curiosity, and to thus strive 
towards their own insights. 
Such student-driven inquiry builds confidence, 
while letting students take ownership of their own work.
Student-driven data collection is one exercise that I've
found particularly useful here; their project ownership is explicit,
and the collected data can serve guide and facilitate student
projects. The intended result is that students develop an investment 
in the course and associated coursework, committing to long-term 
memory not simply facts, but the full process of exploration and 
discovery.

This path of discovery is, necessarily, paved 
with acquired skills, from routine data analysis to 
high-level mathematics or computer programming. 
In addition to in-class exercises with physical props 
and student interaction, I also use short quizzes and surveys 
that emphasize concept and process. Nonetheless, 
successful development of quantitative skills requires 
prolonged individual attention.  To this end, I 
provide periodic, incremental practice problems
that are both structured and exploratory.
I expect individual write-ups, while encouraging (and sometimes 
requiring) group collaboration; such a configuration provides 
individual motivation and accountability, while fostering
peer learning. In this way, I seek to provide a supportive 
environment where student practice is both expected and rewarded. 
Quizzes, surveys, and periodic written assignments
also provide a critical source of feedback, 
helping to identify struggling individuals or 
problematic concepts.

My approach to direct mentorship and graduate student
training is similar, if more demanding.  I do not 
require prerequisite mathematical or analytical skills,
but I do require curiosity, enthusiasm, and engagement.
I work with students to identify their interests 
and goals, and then collaboratively develop a roadmap 
to achieve those goals.

The ``lightbulb experience'' of student insight is 
deeply satisfying to both teacher and student. 
Whether from deep conceptual connections or a conquered
technical hurdle, these insights ``Aha!'' moments
increase student confidence, while generating thirst for more and 
deeper insight.  Just as often, though, deep insight comes 
from a slow accumulation of skills and connections that, 
in turn, informs and is informed by a student's life.  
Ultimately, my goal is to create a supportive environment 
that fosters such personal growth through practice, 
exploration, and discovery.

\end{document}


%To this day, mathematics and computer programming are still too often viewed as stereotypically ``male'' pursuits. This is an area where I feel particularly well-suited to foster an environment of inclusion.  

%cognitive diversity

%Explicitly addressing the universality of this experience can help students connect with each other, as well as give them permission to seek support. 
