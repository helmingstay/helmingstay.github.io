\documentclass[12pt]{article}
\usepackage[left=0.75in,top=0.6in,right=0.75in,bottom=0.6in]{geometry} % Document margins
%\usepackage{url}
\usepackage{hyperref}
\usepackage[numbers, sort&compress]{natbib}
%\usepackage{citeref}  %Not avail on linux.unm.edu; makes page #s in biblio
%\usepackage{graphicx}
%\usepackage{multicol}

\author{Christian Gunning}
\title{Teaching Philosophy and Experience}
%\date{\today}
\begin{document}
\maketitle
\thispagestyle{empty}
\linespread{1.25}\selectfont


My teaching philosophy is meaning-centered and skill-oriented.  Whether teaching
programming to biology graduate students, probability and statistics to biology
under-graduates, or introductory biology to non-major undergraduates, I
prioritize curiosity and exploration.  I cultivate strong analytical skills
through regular assignments that give students the tools and confidence to build
successful academic careers. Ultimately, I seek to develop student ownership of
the academic experience through engagement and self-determination.

In designing curricula, I first seek out material that students can identify with,
whether through popular culture, humor, or common human experience.  I
frequently employ multimedia visual aids, such as XKCD cartoons and segments
from professionally produced video series (e.g., Crash Course Biology).  This
supports students with non-traditional learning styles, and provides
supplemental learning material that students can review at their leisure. In
this way, I aim for immediate and sustained student engagement and lecture 
participation.

To foster student engagement, I also frequently employ
 tangible examples and exploratory ``active learning'' exercises.
In non-major biology classes, for example, a simple 
``mating'' game that uses standard playing cards was met 
with great enthusiasm.  Here, the genome of each ``parent'' 
is a single color, with suites representing grandparent genomes.
Random assortment quickly reveals that offspring always are
evenly matched in color (parent), but differ widely in the 
relative contribution of grandparents.  In more technically
sophisticated classes, such as the {\em Probability for Scientists}
class that I proposed, co-designed, and co-taught at UNM,
I frequently use coin flips and dice rolls to
illustrate topics in randomness (side-note: 
pennies are inexpensive, abundantly available, and endlessly 
amusing teaching aids).
Using coins, students can explore a range 
of complex phenomena, from memory-less Markov chains to the 
central limit theorem.  These physical activities provide
``tangible metaphors'' that are 
accessible to students from a wide range of backgrounds, 
and provide a core vocabulary with which to explore more 
complex topics.

Inquiry-based learning forms a central role in my teaching.
I regularly pose carefully constructed 
questions that challenge students to identify hidden connections. 
One example involves asking students to imagine searching for 
their keys in one dimension,
along a line or curve, versus two dimensions, a surface or
table-top, versus a cluttered house.  Here we see the {\em 
curse of dimensionality} in action, where a hypothetical 
10-dimensional space can be imagined through its search 
complexity.  
I also emphasize collaborative exploration, as well as trial and error.
For example, students can use dice to experimentally explore the 
relationship between variance and sample size. What is the 
probability of rolling ``snake eyes''? Let's find out!
Teams of students will race to conduct experiments, and then 
each team's results are collected and compared.
This sort of tactile, student-centered exploration provides 
a powerful companion to more analytical material, 
helping increase student motivation and recall.

This path of discovery is, necessarily, paved 
with acquired skills, from routine data analysis to 
high-level mathematics or computer programming. 
In addition to in-class exercises with physical props 
and student interaction, I also use frequent short quizzes 
that emphasize concept and process. 
Quantitative skill development also requires 
regular practice. To this end, I regularly provide 
goal-oriented assignments that develop quantitative and 
critical thinking skills through a series of manageable steps. 
I provide clear expectations, and expect individual write-ups, 
while encouraging (and often requiring) group collaboration. 
In this way, I maintain individual accountability, while fostering
peer learning. Quizzes and written assignments
also provide a critical source of feedback, 
helping to identify struggling individuals or 
problematic concepts.

Finally, I challenge students to pose novel questions based on
their own interests and curiosity, and to strive 
towards their own insights. 
Student-driven inquiry builds confidence, 
while letting students take ownership of their own work.
I have successfully used online surveys to facilitate student 
data collection. By encouraging students to collect data that 
is personally interesting and meaningful, they gain ownership
of the process and the product.
The data thus collected can then facilitate student
projects. As a result, students develop a personal investment 
in the course, committing to long-term memory not simply facts, 
but the full process of exploration and discovery.

My approach to direct mentorship and graduate student
training is similar, if more demanding. I do not
require prerequisite mathematical or analytical skills,
but I do require curiosity, enthusiasm, and engagement.
Indeed, I actively solicit students with 
non-traditional backgrounds who have developed 
independence and discipline outside of academia.
I work with students to identify their interests 
and goals, and then collaboratively develop a roadmap 
to achieve those goals.

A large and growing number of STEM students are seeking
access to sophisticated quantitative and computational 
skills. Yet integrating formal training in mathematics,
statistics, and computer programming into existing
curriculums remains a challenge. I excel at  
teaching problem-oriented quantitative skills to 
STEM students from a range of backgrounds and disciplines.
This training provides a solid foundation for success 
in whatever path students chose to pursue, from industry to 
academia.

The ``lightbulb experience'' of student insight is 
deeply satisfying to both teacher and student. 
Whether from deep conceptual connections or a conquered
technical hurdle, these ``Aha!'' moments
increase student confidence, and generate thirst for 
deeper insight.  Just as often, deep insight comes 
from a slow accumulation of skills and connections that, 
in turn, informs and is informed by a student's life.  
Ultimately, my goal is to create a supportive, stimulating,
and challenging environment that fosters personal growth 
through practice, exploration, and discovery.

\end{document}


%To this day, mathematics and computer programming are still too often viewed as stereotypically ``male'' pursuits. This is an area where I feel particularly well-suited to foster an environment of inclusion.  

%cognitive diversity

%Explicitly addressing the universality of this experience can help students connect with each other, as well as give them permission to seek support. 
