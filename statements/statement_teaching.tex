\documentclass[12pt]{article}
\usepackage[left=0.75in,top=0.6in,right=0.75in,bottom=0.6in]{geometry} % Document margins
%\usepackage{url}
\usepackage{hyperref}
\usepackage[numbers, sort&compress]{natbib}
%\usepackage{citeref}  %Not avail on linux.unm.edu; makes page #s in biblio
%\usepackage{graphicx}
%\usepackage{multicol}

\author{Christian Gunning}
\title{Teaching Philosophy and Experience}
%\date{\today}
\begin{document}
\maketitle
\thispagestyle{empty}
\linespread{1.25}\selectfont
%% a statement of activities contributing to diversity, equity and inclusion in Academia

My teaching philosophy is meaning-centered and skill-oriented.
For both students and teachers, meaning begins with recognition.
As such, my first criteria in lesson design
is to find material that students can identify with,
whether through popular culture, humor, or common human experience. 
For example, I often use coin flips to illustrate topics in 
randomness. Coin flips are commonly-experienced simple phenomena
that can be used to illustrate a range of complex phenomena,  
from Markov chains to the central limit theorem.
I employ this comfort of the familiar to build towards insight,
where a previously hidden relationship becomes revealed.
Finally, I challenge students to pose questions based on
their own interests and curiousity, and to strive towards 
their own insights. This student-driven inquiry builds confidence, 
while letting students take ownership of their own work.
The intended result is that students internalize their explorations,
committing to long-term memory not simply facts, but the full
process of exploration and discovery.

This path of exploration and discovery is, necessarily, paved 
with acquired skills, from simple data analysis and synthesis to 
high-level mathematics or computer programming. I seek to facilitate 
skill acquisition by providing frequent, incremental, and exploratory
practice problems. I employ in-class exercises with physical props 
and student interaction, short quizzes and surveys that emphasize 
process over facts, and periodic written assignments.  For most assignments,
I expect individual write-ups, while encouraging (and sometimes requiring) group
collaboration. I find that this configuration fosters peer teaching, while maintaining 
individual motivation and accountability. 
In this way, I seek to provide a supportive environment in which student practice 
is both expected and rewarded. 

The ``lightbulb experience'' of student insight is 
deeply satisfying to both teacher and student. Such insights in turn 
foster deeper exploration and discovery by increasing student confidence, 
and by generating personal desire for deeper insight.
 Such insights can arise from the simple connections of two previously 
unrelated ideas, such as between ``false positives'' 
and humans' propensity to ``see monsters in the dark'',
or from hard-won technical skills. As often, student insight comes not 
in a flash, but in a slow accumulation of skills and connections, 
leading to a greater vision that informs a student's life.  


\end{document}


%To this day, mathematics and computer programming are still too often viewed as stereotypically ``male'' pursuits. This is an area where I feel particularly well-suited to foster an environment of inclusion.  

%cognitive diversity

%Explicitly addressing the universality of this experience can help students connect with each other, as well as give them permission to seek support. 
