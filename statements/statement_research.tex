\documentclass[12pt]{article}
\usepackage[left=0.75in,top=0.6in,right=0.75in,bottom=0.6in]{geometry} % Document margins
%\usepackage{url}
\usepackage[hidelinks]{hyperref}
\usepackage[numbers, sort&compress]{natbib}
\bibliographystyle{unsrt}
%\usepackage{citeref}  %Not avail on linux.unm.edu; makes page #s in biblio
%\usepackage{graphicx}
%\usepackage{multicol}

\author{Christian Gunning}
\title{Quantitative Population Ecology}
%\date{\today}
\begin{document}
%\maketitle
\thispagestyle{empty}
\linespread{1.25}\selectfont
%% a statement of activities contributing to diversity, equity and inclusion in Academia

My research explores the spatiotemporal dynamics 
of real-world populations and metapopulations. 
My primary focus is on discrete, stochastic, and non-linear processes,
including invasion and persistence, and population responses to perturbation.
%while observational processes are often poorly characterized. 
I use mechanistic simulation 
models to {\em predict} dynamics from theory, and statistical models 
to match empirical observations to simulation.
I also use models to estimate and account for incomplete 
observation, which commonly distorts empirical datasets.
Finally, I employ macroecological approaches 
to identify conserved scaling relationships in 
data, which I then test against ecological theory and 
simulation models. Throughout, I seek the causal determinants of
aggregate population behavior using the simplest 
possible explanations.

%, and demographic founder events such
%as colonization and development of resistance. 

To date, my work has focused largely on childhood diseases in the pre-vaccine
era. Human disease ecology provides an attractive study system, leveraging
extensive public health observational infrastructure while fulfilling the scientific
imperative towards public benifit. During my Ph.D.\ work with Helen Wearing
at UNM, I employed seed grant funding to design, implement, and staff a
technology-assisted data entry program that transcribed historical U.S.\ public
health reports. The end result is more than 20 years of weekly measles and
pertussis case reports in 80 cities and 35 states, influenza
case and death reports, and associated demographic records.  Using this
dataset, I have demonstrated the effects of spatial and temoral aggregation on
metapopulation stochastic extinction \citep{gunning2013probabilistic}.
I have shown the obscuring effects of incomplete observation
in disease reporting \citep{gunning2014conserved}, and its close correlation 
with socioeconomic covariates. Combining these estimates of incomplete reporting
with epidemic theory and statistical models, I have shown that cryptic 
disease presence commonly occurs, and is likely affected by disease life
history \citep{gunning2016cryptic}.  
In the future, I plan to characterize the (pre-vaccine era) dependence of influenze
death reports on case reports, allowing for a clearer interpretation 
of historical records. I also plan to collaborate with 
Micaela Martinez, now at Princeton University, to compare the geographical
and socioeconomic determinants of epidemic timing between 
infectious human diseases, including influenza, measles, and pertussis.

Taken together, this work complements modern public 
health research on one hand, and theoretical disease modeling on the other.
The fields of public health and epidemiology are frequently results-oriented, 
seeking immediate solutions to particular problems. 
Theoretical disease modeling, on the other hand, commonly makes simplifying 
assumptions to better focus on particular aspects of ecological theory.
My research aims to bridge this
gap by drawing ecological theory into the public health forum, and by
facilitating a more robust empirical validation of theory-driven disease 
models.

My post-doctoral work, with Fred Gould and Alun Lloyd at NCSU, has focused 
on the population ecology of the {\em Aedes aegypti} mosquito.
{\em Aedes aegypti} is the primary vector of a disturbing array of human
diseases, including dengue, chikungunya, and zika, yet its basic biology remains
poorly characterized. Projecto Dengue is a long-term, multi-agency project
based in Iquitos, Peru that seeks effective vector control strategies. 
Under the direction of Tom Scott, UC Davis, Projecto Dengue has conducted 
a large, multi-year urban field 
trial of indoor insecticide spraying.  In Iquitos, we observe non-linear, 
density-dependent responses of adult mosquito populations to human
perturbations, along with suggestions of emerging insecticide resistence.
This field trial provides a unique window into the urban population ecology of {\em
Aedes}, while raising fundamental questions about the long-term efficacy of 
chemical population control. 

My role in Projecto Dengue has included data management,
where I reconciled complex and sometimes 
contradictory data streams into a cohesive picture of mosquito
population dynamics. In addition, I have led data visualization,
statistical analysis, and manuscript preparation.
The above dataset is also being used to parameterized 
the agent-based simulation model Skeeter Buster
\citep{magori2009skeeter, okamoto2013reduce}. 
Skeeter Buster pre-dates my time in the Gould-Lloyd group; nonetheless,
I have led its continued development, including model validation and
integration of field data.

The rate and intensity of {\em Aedes} migration over urban landscapes 
remains poorly characterized, and yet has wide-ranging human impacts on
control efforts and insecticide resistance prevention.
I plan to leverage the rich Projecto Dengue dataset 
to elucidate landscape level {\em Aedes} population dynamics 
through a collaboration with Fred Gould and Jennifer Baltzegar at NCSU,
Dr. Gould's lab has expertise with modern genotype-by-sequencing (GBS) 
methods in insects (i.e., ddRADseq). We will combine existing 
spatiotemporal records of spraying and insect sample collection with 
GBS-derived haplotype data to infer landscape-level population processes, 
including migration rates census population size, and population responses 
to transient vector control.

At NCSU, I've played an active role in an ongoing collaboration 
that evaluates the performance of vector control strategies employing 
so-called {\em gene drives}. Here, engineered organisms contain genetic 
constructs that cause biased inheritance, driving the construct through
a population.  These organisms can be released to
reduce wild pest populations through high genetic load, or to replace 
wild populations, e.g., with a pathogen-resistant engineered strain. 
The subject of intense scrutiny, gene drives differ 
widely in design and intended performance. Using simple models 
that combine population ecology with population genetics, this collaboration 
aims to identify ecological uncertainties with proposed genetic 
constructs, along with possible reversal options \citep{vella2016evaluating}.

Looking to the future, I see a number of promising ecological research
opportunities in non-equilibrium population dynamics at the University
of Michigan. Lyme disease presence is now increasing throughout the 
mid-West, and yet has come late to Michigan, whose unique
penisular geography appears to have slowed disease invasion \citep{hamer2010invasion}. 
Ongoing monitoring of tick populations at the Edwin S. George Reserve (ESGR)
could reveal a virgin soil invasion of this highly visible human pathogen.
I would welcome a collaboration or graduate student project using ESGR resources 
to explore the invasion biology of {\em Ixodes scapularis}, particularly in 
response to urbaninzation.  Another disruptive ecological invader in Michigan 
is garlic mustard, {\em Alliaria petiolata}. Garlic mustard provides an 
excellent opportunity for citizens to participation in 
population control studies, and to increase spatiotemporal 
monitoring of an ongoing non-native invasion. 
I would also welcome a collaboration that both models and 
monitors the effectiveness of a biological control programs \citep{evans2012importance}.

We live in a world awash with data, from novel remote sensing 
capabilities such as the NSF's National Ecological Observatory 
Network (NEON), to next-generation sequencing technologies.
Yet translating this flood of data into {\em knowledge} 
remains challenging.  One of my great strengths is the
ability to rapidly integrate large, complex, and often messy 
data assemblages into cohesive, archival-quality databases.
I thrive in these data-rich systems, using the best available 
tools to visualize system processes, and then
translate observations into testable hypotheses.
In this way, my work is dominated by an iterative relationship 
between ecological theory and empirical data. 

I use cutting-edge open source computational tools, including 
SQL for data management, the R languages for data visualization
and statistical modeling, and C++ for high-performance simulation
modeling. I am a strong proponent of open science and reproducible 
research methods. Using open source software
and careful documentation practice, I have increased the efficiency 
and overall quality of my work, while enhancing collaborative 
opportunities. A reproducible workflow also facilititates student 
collaboration, facilitating both experimentation and 
accountability. 
Want to know who fixed that bug? Just check the git logs. 
Someone's edits broke the project? No worries, just do 
{\texttt git revert}.

As a quantitative ecologist, I draw 
methodology and metaphor from a range of disciplines, including 
chemistry, physics, probability theory, and computer science.
From undergraduate experience in molecular biology and 
bioinformatics, to master's work combining abiotic 
climactic datasets with ecological observation, I have broad
and detailed understanding of modern biological systems 
theory.

%From Markov processes and ergodic theory to information theory 
%and statistical modeling, I employ a diverse yet robust 
%conceptual toolset to 

\bibliography{statement}

\end{document}
