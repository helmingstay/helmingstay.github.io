\documentclass[12pt]{article}
\usepackage[left=0.75in,top=0.6in,right=0.75in,bottom=0.6in]{geometry} % Document margins
%\usepackage{url}
\usepackage{hyperref}
\usepackage[numbers, sort&compress]{natbib}
\bibliographystyle{abbrvnat}
%\usepackage{citeref}  %Not avail on linux.unm.edu; makes page #s in biblio
%\usepackage{graphicx}
%\usepackage{multicol}

\author{Christian Gunning}
\title{Research Program}
%\date{\today}
\begin{document}
\maketitle
\thispagestyle{empty}
\linespread{1.25}\selectfont
%% a statement of activities contributing to diversity, equity and inclusion in Academia

I combine ecological theory, empirical observation, and 
mechanistic models to explore the processes
and patterns that shape biological populations and metapopulations. 
I am most interested in spatiotemporal dynamics of real-world populations,
particularly discrete, stochastic, and non-linear processes.
To date, my study systems have included human childhood diseases, along
with the {\em Aedes aegypti} mosquito. Using historical observations, along with
modern field trials, I have explored drivers of
stochastic extinction and persistence,
population response to perturbation, and metapopulation synchrony.
Throughout, I seek the causal determinants of aggregate 
system behavior using the simplest possible explanations.
%, and demographic founder events such
%as colonization and development of resistance. 

Ecological systems are inherently non-linear and stochastic,
while observational processes are often poorly characterized. 
To identify deterministic needles in these haystacks of 
entangled processes and ``iterated noise'',
I use simple mechanistic simulation models to 
{\em predict} dynamics from theory, and probabilistic 
statistical models to match observation to simulation.
I also use macroecological approaches to identify 
conserved invariant scaling relationships between 
observables, along with the theoretical drivers of these
relationships.

To date, much of my work has focused on childhood diseases 
in the pre-vaccine era. Human disease ecology provides an 
attractive study system, 
leveraging public health observational infrastructure while 
fulfilling the scientific imperative towards public benifit.
As part of my Ph.D. work at the University of New Mexico with
Dr. Helen Wearing, I designed and implemented a 
technology-assisted data entry program to transcribe historical U.S. 
public health reports. The end result is more than 
20 years of weekly measles and
pertussis case reports in some 80 cities and 35 
states, along with influenza case and death reports.
I have used this dataset, along with detailed demographic records, to draw 
novel ecological insight into disease processes. From the 
effects of spatial and temoral aggregation on metapopulation stochastic
extinction \citep{gunning2013probabilistic}, to the effects of incomplete
observation \citep{gunning2014conserved}, this system
has yielded powerful insights into metapopulation 
ecology and human health.
Future projects in this system include a study of 
the geographical and socioeconomic determinants of 
the epidemic timing among populations 
using wavelet analysis.  While previous research has 
suggested spatial waves of pertussis incidence 
in the U.S. \citep{choisy2012changing}, as well as 
as modern socioeconomic roles in influenza spread \citep{viboud2006synchrony}, 
a detailed comparison between diseases in a single 
metapopulation is lacking.

My post-doctoral work has focused on the 
population ecology and vector biology of the 
{\em Aedes aegypti} mosquito,
with Drs. Fred Gould and Alun Lloyd at 
North Carolina State University. I am an
active participant in the large, multi-agency
Projecto Dengue in Iquitos, Peru, led by Dr.
Tom Scott of UC Davis. Here, I 
have provided leadership in reconciling complex data 
streams from spray control field trials, as well as
in statistical analysis and manuscript preparation.
In this system, we see the emergence 
of insecticide resistence, along with signatures of  
non-linear, density-dependent feedback resulting 
from human perturbation. I also plan on further  
collaboration with Dr. Gould's graduate student 
Jennifer Baltzegar to use modern genotype by 
sequencing methods to explore patterns of local 
migration, census population size, and the 
population response to perturbation by human 
vector control. 

We live in a world awash with data. From the remote sensing 
facilities of the National Science Foundation's National 
Ecological Observatory Network, to next-generation 
sequencing technologies, to citizen science initiatives 
such as the Cornell Lab of Ornithology's eBird,
this is a golden era of ecological observation. 
Yet translating this flood of data into {\em knowledge} 
remains challenging.  One of my great strengths is the
efficient assimilation of large multidimensional data 
assemblages into a cohesive whole. 
I use cutting-edge quantitative tools, including 
SQL for data management, the R languages for data visualization,
and statistical modeling, and C++ for high-performance simulation
modeling. This pipeline allow me to rapidly explore novel data,
and to then translate observations into testable hypotheses.


At NCSU, I've also built ongoing mentoring and 
collaborative relationships with several graduate students,
using models combining population dynamics with 
populaiton genetics models to 
understand the consequences novel gene drive strategies. 
Here, engineered organisms are released as part of vector 
control campaigns. Our work uses simple models to identify
ecological uncertainties with proposed genetic constructs
and associated release plans.



%As a quantitative ecologist, I draw 
%methodology and metaphor from a range of disciplines, including 
%chemistry, physics, probability theory, and computer science.
%From Markov processes and ergodic theory to information theory 
%and statistical modeling, I employ a diverse yet robust 
%conceptual toolset to 

My involvement with the Santa Fe Institute has provided
me with  
Interdisciplinary collaboration relies upon a firm
foundation in one's own area of expertise, along
with a curiousity, trust, and willingness to learn
from other experts.

, information-theoretic 

Finally, I am a strong proponent in open, 
reproducible research. 
I use reproducible research best practices 
to maximize work efficiency and quality, and increase
the ability for others to quickly and faithfully 
build upon my work, thus increasing its impact
on society at large.

\bibliography{statement}

\end{document}
