\documentclass[12pt]{article}
%\usepackage{url}
\usepackage{hyperref}
\usepackage[numbers, sort&compress]{natbib}
%\usepackage{citeref}  %Not avail on linux.unm.edu; makes page #s in biblio
%\usepackage{graphicx}
%\usepackage{multicol}

\author{Christian Gunning}
\title{}
\date{\today}
\begin{document}

I use simple mechanistic models and modern statistical inference 
methods to explore natural dynamical systems.
To date, the central focus of my research has been the interface 
between deterministic and stochastic dynamics. 
An organizing principle of my research is to employ both data and models 
together.  
My goal is to use the simplest mechanistic models and inferences 
method available to identify causal determinants of aggregate 
system behavior.

To date, my research has focused primarily on biotic populations
of organisms that cause or transmit human diseases.  My work has 
encompassed a range of natural systems, 
including human microparasites such as measles,
and disease vectors such as the mosquito 
\em{Aedes aegypti}.  In addition, I have extensive 
experience incorporating abiotic dynamics such as weather
and hydrology.

Natural populations are irreducibly composed of 
organisms that exhibit a wide range of individual variation 
and stochasticity.  This combination of discreteness and 
stochasticity can have important dynamical consequences in
small populations. In particular, both extinction and migration
can hinge on the behaviors or outcomes of a very small
number of individuals.
Even so, as \em{processes}, extinction and migration are best
described in aggregate as population-level outcomes. I employ 
probabilistic, distributional approaches to bridge this
gap between the particular (individual-level) and 
the general (population-level).

Natural abiotic systems exhibit stochasticity and 
autocorrelation on a wide range of scales that, in turn,
influence biotic systems. From the simple influences of 
seasonality to more complex dynamics such as growth, 
maturation, and reproduction, the 

Choosing an appropriate level of mechanistic modeling is a key 
challenge in all sciences.  Here, for example, a
deterministic model such as a continuous-time 
2-species predator-prey model (analogous to a frictionless pendulum) 
neglects biological realism in favor of 
analytic tractablility.  On the other hand, a highly detailed 
and potentially more realistic agent-based model
can be inherently too complex to identify mechanistic links 
between cause and effect.  My work favors an intermediate 
level of modeling that explicitly includes stochasticity and 
discreteness, while favoring memoryless Markovian dynamics of 
classic differential equations models over  
more complex state-dependent dynamics.

Elucidating interactions between the individual-level and 
population-level is a fertile area of scientific 
inquiry that has been greatly facilitated by modern tools.
Novel computational methods and hardware ease model 
development and simulation, while citizen science, remote
sensing tools, and novel genomic methods have accelerated
data collection. As such, a rich trove of both data and 
tools are now available to explore pressing modern concerns
in population dynamics, including disease elimination, 
pest control, native species extinction, 
and non-native species 
invasion.  Consequently, the ability to rapidly 
integrate novel datasets and mechanistic simulations 
is more important now than ever before.

\section*{Tools}
In my role as a data-driven researcher, I rely on time-tested
computational tools such as SQL and R to rapidly organize and visualize
large quantities of data.  This allows me to rapidly construct statistical
models, while integration of R and C++ allows me to quickly construct and 
assess mechanistic models.  

\section*{Summary}
The overarching goal of my work is to unveil the causal 
determinants of each system's aggregate behavior, which in
turn is an interplay between randomness and determinism.
My work draws inspiration both from Robert May, who popularized  
the fact that ``simple models can display complex behaviors'', 
and from macroecology and information theory, which describe how 
predictable, low-dimensional patterns can emerge from the 
aggregate dynamics of complex systems.  I seek to reconcile
these approaches: I use simple probabilistic metrics to 
describe natural systems, and use simple mechanistic 
models to explain their complex behaviors.


%\maketitle

\section{}

%\bibliographystyle{plainnat}
	% plain.bst included standard in latex
%\bibliography{$DOCHOME/BIB/bemp}
	% sample.bib is user-created biblio info
	% use \citep or \citet*, etc.


\end{document}


	%\begin{multicols}{2}
    % \tableofcontents
    % \\ Newline, \\* Newline-without-pagebreak
    % \newpage \pagebreak paginates
    % \mbox prohibits linebreak, \fbox same w/visible box
    % `` these are quotes '' and \@. this ends an uppercase sentence
    % - -- --- are different
    % Text styles \emph{} or {\em ... }, {\small  }, {\large  }
    % \begin{quote} \end{quote}
%-----Example Figure Follows
%\begin{figure}
%  \begin{center}
%    \includegraphics[width=5.5in]{eps/N-layout-fin.eps}
%  \end{center}
%  \caption{\small Overview of North Site 1km Buffers with Dams and USGS River Guages}
%  \label{fig-overview}
%\end{figure}
