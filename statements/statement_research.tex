\documentclass[12pt]{article}
%\usepackage{url}
\usepackage{hyperref}
\usepackage[numbers, sort&compress]{natbib}
%\usepackage{citeref}  %Not avail on linux.unm.edu; makes page #s in biblio
%\usepackage{graphicx}
%\usepackage{multicol}

\author{Christian Gunning}
\title{}
\date{\today}
\begin{document}

My research employs simple mechanistic models and modern statistical inference 
methods to reveal hidden patterns in natural dynamical systems.
An organizing principle of my work is the integration of empirical 
data with both statistical and mechanistic models.
My over-arching goal is to identify causal determinants of aggregate 
system behavior using the simplest possible mechanistic models and 
inferences methods. 

To date, I have focused primarily on populations
of organisms that cause or transmit human diseases.
I have studied a range of natural systems, 
including human microparasites such as measles,
and disease vectors such as the mosquito 
\em{Aedes aegypti}.  In addition, I have extensive 
experience incorporating abiotic dynamics such as weather
and hydrology.


At present, my core focus is on the interface 
between deterministic and stochastic dynamics, 
and between continuous and discrete systems.
Natural populations are irreducibly composed of 
organisms that exhibit a wide range of individual variation 
and stochasticity.  This combination of discreteness and 
stochasticity can have important dynamical consequences in
small populations. In particular, both extinction and migration
can hinge on the behaviors or outcomes of a very small
number of individuals.
Yet as aggregate \em{processes}, extinction and migration 
are most parsimoniously described as population-level 
outcomes. I employ
probabilistic, distributional approaches to bridge 
observations of the particular (individual-level) with 
descriptions of the general (population-level).

Biotic systems are also affected by 
abiotic environments, which themselves
exhibit stochasticity and 
autocorrelation over a wide range of scales.
Seasonality, for example, causes 
yearly changes in temperature and sunlight
over large spatial areas, while precipitation 
commonly shows substantial clustering in 
both space and time.  
Widespread environmental variation provides both
opportunity and challenge: opportunity in the form
of natural experiments, and challenge in the form
of increased complexity and added uncertainty.  
Indeed, distinguishing exogenous forcing from 
endogenous dynamics is a long-standing challenge 
in population dynamics.

Choosing an appropriate level of mechanistic modeling is a key 
challenge in all sciences.  In natural populations, for example, a
a continuous-time deterministic
2-species predator-prey model (analogous to a frictionless pendulum) 
neglects biological realism in favor of 
analytic tractablility.  On the other hand, highly detailed 
and potentially more realistic agent-based models
quickly grow too complex to identify mechanistic links 
between cause and effect.  

Broadly speaking, I prefer parsimonious ``strategic'' models 
over kitchen-sink ``tactical'' models. My work favors an intermediate 
level of modeling that explicitly includes stochasticity and 
discreteness, while favoring memoryless Markovian dynamics of 
classic differential equations models over  
more complex state-dependent dynamics. 

\section*{Tools}
In my role as a data-driven researcher, I rely on time-tested
computational tools. The R programming language provides me with a
unifying core for data assimilation, visualization, statistical
modeling, simulation prototyping, and reproducible manuscript 
preparation.   
I use SQL (and Postgresql in particular) for data management, 
including complex geospatial GIS data. To the extent possible, I  
employ data management best-practices to facilitate reproducibility,
interoperability, and scalability.

I commonly use R for (generalized) linear modeling, as well as 
parametric and non-parametric bootstrapping. I also 
employ time series analysis tools, including Fourier 
and wavelet transforms and ARIMA models.

I use both R and C++ to rapidly construct, simulate, and
analyze mechanistic models. Common model structures include 
discrete formulations of compartment epidemiological models 
(``SIR'' models) that are coupled into metapopulations.

Altogether, modern computational tools allow me to
assimilate large and complex datasets, simulate mechanistic
models, and conduct inference with great efficiency and fidelity.

\section*{Summary}
Inexorable technological development
in computation and data acquisition
provides increasingly powerful tools to
elucidate interactions between individual-level
population-level dynamics. 
Novel computational hardware and methods ease model 
development and simulation, while citizen science, remote
sensing tools, and novel genomic methods have accelerated
data collection. As such, a rich trove of both data and 
tools are now available to explore pressing modern concerns
in population dynamics, including disease elimination, 
pest control, native species extinction, 
and non-native species 
invasion.  Consequently, the ability to rapidly 
integrate novel datasets and mechanistic simulations 
is more important now than ever before.

The overarching goal of my work is to unveil the causal 
determinants of each system's aggregate behavior, which in
turn is an interplay between randomness and determinism.
My work draws inspiration from Robert May, who popularized  
the fact that ``simple models can display complex behaviors'', 
and from macroecology and information theory, which describe how 
predictable, low-dimensional patterns can emerge from the 
aggregate dynamics of complex systems.  I seek to reconcile
these approaches: I use simple probabilistic metrics to 
describe natural systems, and simple mechanistic 
models to explain their complex behaviors.


%\maketitle

\section{}

%\bibliographystyle{plainnat}
	% plain.bst included standard in latex
%\bibliography{$DOCHOME/BIB/bemp}
	% sample.bib is user-created biblio info
	% use \citep or \citet*, etc.


\end{document}


	%\begin{multicols}{2}
    % \tableofcontents
    % \\ Newline, \\* Newline-without-pagebreak
    % \newpage \pagebreak paginates
    % \mbox prohibits linebreak, \fbox same w/visible box
    % `` these are quotes '' and \@. this ends an uppercase sentence
    % - -- --- are different
    % Text styles \emph{} or {\em ... }, {\small  }, {\large  }
    % \begin{quote} \end{quote}
%-----Example Figure Follows
%\begin{figure}
%  \begin{center}
%    \includegraphics[width=5.5in]{eps/N-layout-fin.eps}
%  \end{center}
%  \caption{\small Overview of North Site 1km Buffers with Dams and USGS River Guages}
%  \label{fig-overview}
%\end{figure}
