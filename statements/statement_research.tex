\documentclass[12pt]{article}
\usepackage[left=0.75in,top=0.6in,right=0.75in,bottom=0.6in]{geometry} % Document margins
%\usepackage{url}
\usepackage{hyperref}
\usepackage[numbers, sort&compress]{natbib}
\bibliographystyle{abbrvnat}
%\usepackage{citeref}  %Not avail on linux.unm.edu; makes page #s in biblio
%\usepackage{graphicx}
%\usepackage{multicol}

\author{Christian Gunning}
\title{Research Program}
%\date{\today}
\begin{document}
\maketitle
\thispagestyle{empty}
\linespread{1.25}\selectfont
%% a statement of activities contributing to diversity, equity and inclusion in Academia

I combine ecological theory and empirical observation, along with mechanistic
and statistical models, to explore the processes shaping real-world biological
populations and metapopulations.  My primary focus is on the spatiotemporal
dynamics of discrete, stochastic, and non-linear processes.  My study systems
have included human childhood diseases, along with the {\em Aedes aegypti}
mosquito.  Using historical observations and modern field trials, my work
explores drivers of stochastic extinction and persistence, metapopulation
synchrony, and population responses to perturbation.  Throughout, I seek the
causal determinants of aggregate system behavior using the simplest possible
explanations.
%, and demographic founder events such
%as colonization and development of resistance. 

Ecological systems are inherently non-linear and stochastic, while observational
processes are often poorly characterized.  To identify deterministic needles in
these haystacks of entangled processes and ``iterated noise'', I use simple
mechanistic simulation models to {\em predict} dynamics from theory, and
probabilistic statistical models to match observation to simulation.  I also use
macroecological approaches to identify conserved invariant scaling relationships
between observables, along with the theoretical drivers of these relationships.

To date, much of my work has focused on childhood diseases in the pre-vaccine
era. Human disease ecology provides an attractive study system, leveraging
public health observational infrastructure while fulfilling the scientific
imperative towards public benifit.  During my Ph.D.\ work with Dr. Helen Wearing
at UNM, I employed seed grant funding to design, implement, and staff a
technology-assisted data entry program that transcribed historical U.S.\ public
health reports. The end result is more than 20 years of weekly measles and
pertussis case reports in some 80 cities and 35 states, along with influenza
case and death reports and associated demographic records.  I have used this
dataset to demonstrate the effects of spatial and temoral aggregation on
metapopulation stochastic extinction \citep{gunning2013probabilistic}. I have
also demonstrated the obscuring effects of incomplete observation
\citep{gunning2014conserved}, which can aid further modern disease control
efforts.  Future projects in this system include a study of the relation between
pre-vaccine era influenze case and death reports, allowing for a better
leveraging of historical records. In addition, a planned collaboration with Dr.
Micaela E. Martinez, now at Princeton University, will compare the geographical
and socioeconomic determinants of epidemic timing between a range of acutely
infectious human diseases, including influenza, measles, and pertussis.

I am a strong proponent of open science and reproducible 
research methods, and I find that using open source software
tools and careful documentation increases the efficiency 
and quality of my work. It also enhances collaborative 
opportunities by allowing others to quickly and faithfully 
assimilate, reproduce and participate in my research. 
A reproducible workflow also facilititates student 
collaboration, encouraging individual contribution and 
experimentation along with personal accountability. 
Want to know who fixed that ugly bug? Just check the git logs. 
Someone's edits broke the project? {\texttt git revert}.

My post-doctoral work has focused on the population ecology and vector biology
of the {\em Aedes aegypti} mosquito with Drs. Fred Gould and Alun Lloyd at NCSU.
{\em Aedes aegypti} is the primary vector of a disturbing array of human
diseases, including dengue, chikungunya, and zika, yet its basic biology remains
poorly characterized. Along with Drs.  Gould and Lloyd and Dr. Amy Morrison, I
have played an active role in the large, multi-agency Projecto Dengue in
Iquitos, Peru, led by Dr. Tom Scott of UC Davis.  In pursuit of effective {\em
Aedes aegypti} control measures, Projecto Dengue has conducted several
large-scale urban field trials, including indoor insecticide spraying.  These
field trials provide a unique window into the population ecology of {\em Aedes}.
In this system, we observed the emergence of insecticide resistence, along with
signatures of  non-linear, density-dependent feedback resulting from human
perturbation.  

During this project, I have provided valuable leadership 
in field trial data management, reconciling complex and sometimes 
contradictory data streams into a cohesive picture of mosquito
population dynamics.  Using this dataset, I have led hypothesis design,
statistical analysis, and manuscript preparation.
This dataset is also being used to parameterized 
a detailed, agent-based simulation model Skeeter Buster
\citep{magori2009skeeter, okamoto2013reduce}. 
While Skeeter Buster long pre-dates my time in the Gould-Lloyd group,
I have led its continued development, including model validation and
integration of field data.  

These field trials provide a uniquely detailed glimpse 
into {\em Aedes} population dynamics at a landscape level. In particular, 
the rate and intensity of {\em Aedes} migration over urban landscapes 
remains poorly characterized, and yet has wide-ranging effects, from
response to control efforts to establishment of insecticide resistance. I
plan to collaborate with Dr. Gould's graduate student, Jennifer Baltzegar, in
using modern genotype-by-sequencing methods to characterize rates of local
migration, census population size, and population responses to transient 
vector control. In this way, we will combine detailed spatiotemporal 
records about tissue samples with cost-effective next-generation sequencing
methods to infer landscape-level population processes.

At NCSU, I've also played an active role in an ongoing group collaboration 
that uses simulation models combining population
dynamics with population genetics. This project aims to
characterize the performance of vector control strategies that employ so-called
``gene drives''.  Here, engineered organisms are released to reduce wild pest
populations through high genetic load, or to replace wild populations with a
pathogen-resistant strain.  The subject of intense modern scrutiny, gene drives
differ widely in design and intended performance.  Our work uses simple models
to identify ecological uncertainties with proposed genetic constructs and
associated release plans.


In all pursuits, my work is dominated by an iterative relationship 
between ecological theory and empirical data. Rather 
than focusing on a single system, I employ a common
approach to data-rich systems.  Indeed,
we live in a world awash with data. From the remote sensing 
facilities of the National Science Foundation's National 
Ecological Observatory Network, to next-generation 
sequencing technologies, to citizen science initiatives 
such as the Cornell Lab of Ornithology's eBird,
this is a golden era of ecological observation. 
Yet translating this flood of data into {\em knowledge} 
remains challenging.  One of my great strengths is the
efficient assimilation of large multidimensional data 
assemblages into a cohesive whole. 
I use cutting-edge quantitative tools, including 
SQL for data management, the R languages for data visualization,
and statistical modeling, and C++ for high-performance simulation
modeling. This pipeline allow me to rapidly explore novel data,
and to then translate observations into testable hypotheses.

Looking to the future, I see a number of promising research
opportunities in non-equilibrium population ecology. Lyme disease 
presence is increasing throughout the mid-West, and yet has come 
late to Michigan, whose unique
penisular geography has potentially slowing disease invasion \citep{hamer2010invasion}. 
As such, researchers at the University of Michigan have a front-row seat 
for a virgin soil invasion of a highly visible human pathogen.
With this in mind, I plan to pursue a collaborative and/or graduate student led
project on the invasion biology of {\em Ixodes scapularis} and Lyme disease
prevalence.  The highly visible and much-discussed invasion of non-native
garlic mustard, {\em Alliaria petiolata}, provides an excellent opportunity 
for citizen science participation in population control studies.  
Government outreach provides public guidance on non-weed control efforts, 
yet are citizen efforts effective? This system provides an excellent 
test case for the role public outreach can play in shaping local 
ecology, while increasing spatiotemporal monitoring of an ongoing non-native 
invasion. I would also look forward to a collaboration modeling and 
monitoring the effectiveness of biological control programs \citep{evans2012importance}.

As a quantitative ecologist, I draw 
methodology and metaphor from a range of disciplines, including 
chemistry, physics, probability theory, and computer science.
From undergraduate experience in molecular biology and 
bioinformatics, to master's work combining abiotic 
climactic datasets with ecological observation, I have broad
and detailed understanding of modern biological systems 
theory.


%From Markov processes and ergodic theory to information theory 
%and statistical modeling, I employ a diverse yet robust 
%conceptual toolset to 




, information-theoretic 


\bibliography{statement}

\end{document}
