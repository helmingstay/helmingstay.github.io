\documentclass[12pt]{article}
%\usepackage{url}
\usepackage{hyperref}
\usepackage[numbers, sort&compress]{natbib}
%\usepackage{citeref}  %Not avail on linux.unm.edu; makes page #s in biblio
%\usepackage{graphicx}
%\usepackage{multicol}

\author{Christian Gunning}
\title{}
\date{\today}
\begin{document}

I use simple mechanistic models and modern statistical inference 
methods to explore dynamical systems in nature, particularly 
populations of organisms. 
Natural populations are irreducibly composed of 
individual organisms, while often conserved dynamics best 
described in aggregate, such as population growth and migration.  
A central focus of my 
research is this interface between deterministic
mean-field dynamics and fully stochastic individual-based dynamics.  
Simple mechanistic population models, such as continuous-time 2-species 
predator-prey models, neglect biological realism in favor of 
analytic tractablility, while agent-based models often   
obscure mechanistic relationships between cause and effect.

An organizing principle of my work is to employ both data and models 
together. In my role as a data-driven researcher, I rely on time-tested
computational tools such as SQL and R to rapidly organize and visualize
large quantities of data.  This allows me to rapidly construct statistical
models, while integration of R and C++ allows me to quickly construct and 
assess mechanistic models.  

My goal work is to use the simplest model or method available that 
faithfully represents the system at hand.  

My research encompasses a range of natural systems and methods.  

The overarching goal of my work is, of course, to unveil the causal 
determinants of each system's aggregate behavior.  Sometimes, as 
with disease persistence and extinction, the 
effects of stochasticity and discreteness dominate.

My work draws inspiration both from Robert May, who popularized  
the fact that ``simple models can display complex behaviors'', 
and from macroecology and information theory, which describe how 
predictable, low-dimensional patterns can emerge from the 
aggregate dynamics of complex systems.  

Example mechanistic models include coupled 
systems of epidemiologic compartment (``SIR'') models




%\maketitle

\section{}

%\bibliographystyle{plainnat}
	% plain.bst included standard in latex
%\bibliography{$DOCHOME/BIB/bemp}
	% sample.bib is user-created biblio info
	% use \citep or \citet*, etc.


\end{document}


	%\begin{multicols}{2}
    % \tableofcontents
    % \\ Newline, \\* Newline-without-pagebreak
    % \newpage \pagebreak paginates
    % \mbox prohibits linebreak, \fbox same w/visible box
    % `` these are quotes '' and \@. this ends an uppercase sentence
    % - -- --- are different
    % Text styles \emph{} or {\em ... }, {\small  }, {\large  }
    % \begin{quote} \end{quote}
%-----Example Figure Follows
%\begin{figure}
%  \begin{center}
%    \includegraphics[width=5.5in]{eps/N-layout-fin.eps}
%  \end{center}
%  \caption{\small Overview of North Site 1km Buffers with Dams and USGS River Guages}
%  \label{fig-overview}
%\end{figure}
