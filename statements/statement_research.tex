\documentclass[12pt]{article}
\usepackage[left=0.75in,top=0.6in,right=0.75in,bottom=0.6in]{geometry} % Document margins
%\usepackage{url}
\usepackage{hyperref}
\usepackage[numbers, sort&compress]{natbib}
\bibliographystyle{abbrvnat}
%\usepackage{citeref}  %Not avail on linux.unm.edu; makes page #s in biblio
%\usepackage{graphicx}
%\usepackage{multicol}

\author{Christian Gunning}
\title{Research Program}
%\date{\today}
\begin{document}
\maketitle
\thispagestyle{empty}
\linespread{1.25}\selectfont
%% a statement of activities contributing to diversity, equity and inclusion in Academia

My research combines ecological theory, empirical observation, and 
mechanistic models to arrive at unique insights into the processes 
and patterns that shape biological populations and metapopulations. 
My work explores the spatiotemporal dynamics of real-world populations,
focusing on processes that are discrete, stochastic, and non-linear. 
My core areas of study include stochastic extinction and persistence, 
population response to perturbation, and metapopulation synchrony.
My over-arching goal is to identify causal determinants of aggregate 
system behavior using the simplest mechanistic modelspossible.
%, and demographic founder events such
%as colonization and development of resistance. 

We live in a world awash with data, a golden era
of ecological observation, from the remote sensing 
facilities of the National Science Foundation's National 
Ecological Observatory Network, to next-generation sequencing technologies, 
to citizen science initiatives such as the Cornell Lab of 
Ornithology's eBird. 
Indeed, one of my great strengths is the ability to 
efficiently and reliably assimilate complex
assemblages of empirical data into a cohesive whole.
Yet translating this flood of data into {\em knowledge} remains challenging. Ecological
systems are inherently non-linear and stochastic, 
while observational processes are often poorly characterized. 
My approach to this dilema 
employs simple mechanistic simulation 
models to {\em predict} dynamics from theory, 
and probabilistic statistical models to match 
observation to simulation.
I also use macroecological, information-theoretic 
approaches to identify deterministic needles in 
these haystacks of entangled processes 
and ``iterated noise''. 

Much of my work to-date has focused on the childhood diseases 
in pre-vaccine era U.S.\ cities and 
states.  Human disease ecology provides an 
attractive study system, 
leveraging public health observational 
infrastructure while 
fulfilling the scientific imperative towards public benifit.
As part of my Ph.D., I designed and coordinated a technology-assisted
data entry program to transcribe historical U.S. 
public health reports. The end result is more than 
20 years of weekly measles and
pertussis case reports in some 80 cities and 35 
states, along with influenza case and death reports.
I have used this dataset, along with detailed demographic records, to draw novel ecological insight into 
disease processes. From the 
effects of spatial and temoral aggregation on metapopulation stochastic
extinction \citep{gunning2013probabilistic}, to the 
effects of incomplete
observation \citep{gunning2014conserved}, this system
has yielded powerful insights into metapopulation 
ecology and human health.
Future projects in this system include a study of 
the geographical and socioeconomic determinants of 
the epidemic timing among populations 
using wavelet analysis.  While previous research has 
suggested spatial waves of pertussis incidence 
in the U.S. \citep{choisy2012changing}, as well as 
as modern socioeconomic roles in influenza spread \citep{viboud2006synchrony}, a detailed
comparison between diseases in a single 
metapopulation is lacking.

In my post-doctoral studies, I have leveraged my 
data analytic and modelling expertise towards
vector biology as part of the large, multi-agency
Projecto Dengue in Iquitos, Peru. 
I have also, in collaboration with lab graduate
students, explored models
combining population dynamics with genetics to 
understand, for example, the consequences of 
possible insect release strategies as part of 
vector control campaigns.  




%As a quantitative ecologist, I draw 
%methodology and metaphor from a range of disciplines, including 
%chemistry, physics, probability theory, and computer science.
%From Markov processes and ergodic theory to information theory 
%and statistical modeling, I employ a diverse yet robust 
%conceptual toolset to 

Interdisciplinary collaboration relies upon a firm
foundation in one's own area of expertise, along
with a curiousity, trust, and willingness to learn
from other experts.

My research is facilitated by cutting-edge quantitative tools:
the R and C++ programming languages for data visualization,
numerical simulation, and statistical modeling, as well as SQL for 
advanced data management. Through my ongoing 
collaboration with professional software developers,
I have learned the power of software
development industry best practices \citep{spolsky2000joel}.

Finally, I am a strong proponent in open, 
reproducible research. 
I use reproducible research best practices 
to maximize work efficiency and quality, and increase
the ability for others to quickly and faithfully 
build upon my work, thus increasing its impact
on society at large.

\bibliography{statement}

\end{document}
