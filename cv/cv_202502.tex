%%%%%%%%%%%%%%%%%%%%%%%%%%%%%%%%%%%%%%%%%
% Medium Length Professional CV
% LaTeX Template
% Version 2.0 (8/5/13)
%
% This template has been downloaded from:
% http://www.LaTeXTemplates.com
%
% Original author:
% Trey Hunner (http://www.treyhunner.com/)
%
% Important note:
% This template requires the resume.cls file to be in the same directory as the
% .tex file. The resume.cls file provides the resume style used for structuring the
% document.
%
%%%%%%%%%%%%%%%%%%%%%%%%%%%%%%%%%%%%%%%%%

%----------------------------------------------------------------------------------------
%	PACKAGES AND OTHER DOCUMENT CONFIGURATIONS
%----------------------------------------------------------------------------------------

\documentclass{resume} % Use the custom resume.cls style

\usepackage[left=0.75in,top=0.6in,right=0.75in,bottom=0.6in]{geometry} % Document margins
%% footer
\usepackage{fancyhdr}
\usepackage{lastpage}
\usepackage[hidelinks]{hyperref}
\pagestyle{fancy}
\fancyhf{} 
\renewcommand{\headrulewidth}{0pt}
\rfoot{C.E. Gunning 2025-02: Page \thepage \hspace{1pt} of \pageref{LastPage}}
% clear old, activate new footer
%\pagestyle{fancy}

\name{Christian E. Gunning} % Your name
\address{(706)~224-7627 \\ www.x14n.org \\ research@x14n.org} % Your phone number and email
%\address{307 A Princeton SE \\ Albuquerque, NM 87106} % Your address
%Updated \todoy

\begin{document}

%----------------------------------------------------------------------------------------
%	EDUCATION SECTION
%----------------------------------------------------------------------------------------

\begin{rSection}{Education}
{\bf University of New Mexico, Albuquerque} \hfill {\em Summer 2014} \\ 
Ph.D. with Distinction in Biology (Disease Ecology with concentration in
Integrative Biology) \\
Committee: Drs. Helen J. Wearing (advisor), James H. Brown, Melanie Moses, and Erik Erhardt\\
Title: Population and metapopulation ecology of childhood diseases in the pre-vaccine era United States

{\bf University of New Mexico, Albuquerque} \hfill {\em Fall 2009} \\ 
Masters of Water Resources (Riparian Hydroecology) \\
Advisors: Drs. Bruce Thomson and Roy Jemison \\
Title: Estimating phreatophyte evapotranspiration from diel groundwater fluctuations in the Middle Rio Grande Bosque

{\bf University of Georgia, Athens} \hfill {\em Fall 2001} \\ 
Bachelor of Science, Biochemistry and Molecular Biology \\
Advisor: Dr. James Omichinski
\end{rSection}

\newcommand{\authSelf}{{\bf C.E. Gunning}}

%\begin{rSection}{Pending Publications}
%Mentee co-author: $\dagger$

%G. Zilnik$\dagger$, {\bf C.E. Gunning}, F. Gould.
%The Evolution of Fitness Modifiers and Their Impact on Insecticide Resistance.
%In Prep. Draft available on request.

%J. Nightingale$\dagger$, M. Holstad, {\bf C.E. Gunning}.
%Temperature-dependence in sewer blockage frequency. 
%In Prep. Draft available
%on request.
%\end{rSection}

\begin{rSection}{References}
\begin{rSubsection}{Pejman Rohani (\url{rohani@uga.edu})}{}{Odum School of Ecology}{University of Georgia}
\item Co-author and post-doctoral advisor
\end{rSubsection}
\begin{rSubsection}{Christopher J. Gill (\url{Christopher.gill@gatesfoundation.org})}{}{Global Health}{Gates Foundation}
\item Collaborator and co-author (formerly at Boston University School of Public Health)
\end{rSubsection}
\begin{rSubsection}{Alun Lloyd (\url{alun\_lloyd@ncsu.edu})}{}{Department of Mathematics}{North Carolina State University}
\item Co-author and post-doctoral advisor
\end{rSubsection}
\end{rSection}

\begin{rSection}{Peer-Reviewed Publications}
Full list available at \url{https://scholar.google.com/citations?user=y355pzMAAAAJ} 

Mentee co-author: $\dagger$ \quad Co-first author: $\star$


\authSelf, C.J. Gill, P. Rohani.
Shining light on the dark matter of pertussis: evidence for an asymptomatic carriage state from a longitudinal cohort of mother/infant dyads 
(In Review).

A.T. Siceloff, D. Waltman , \authSelf, S.P. Nolan, P. Rohani, N.W. Shariat.
Longitudinal study highlights patterns of Salmonella serovar co-occurrence and exclusion in commercial poultry production 
(In Review).

\authSelf, P. Rohani, L. Mwananyanda, G. Kwenda, Z. Mupila, C.J. Gill.
Young Zambian infants with symptomatic RSV and pertussis infections are frequently prescribed inappropriate antibiotics: a retrospective analysis (2023).
PeerJ.

\authSelf, A. Morrison, K. Okamoto, T.W. Scott, H. Astete, F. Gould,
A.L. Lloyd. A critical assessment of the detailed Aedes aegypti simulation model Skeeter Buster 2 using field experiments of indoor insecticidal control in Iquitos, Peru (2022).
PLOS Neglected Tropical Diseases.

J. Baltzegar, M. Vella$^\dagger$, \authSelf, G. Vasquez, H. Astete, F. Stell, M. Fisher, T.W. Scott, A. Lenhart, A.L. Lloyd, A. Morrison, F. Gould (2021). Evolutionary Applications.

C.J. Gill, \authSelf$^\star$, W.B. MacLeod, L. Mwananyanda, D.M. Thea, R.C. Pieciak, G. Kwenda, Z. Mupila, P. Rohani (2021).  Asymptomatic Bordetella pertussis infections in a longitudinal cohort of young African infants and their mothers. eLife.

R.K. Borchering, \authSelf, Deven V Gokhale$^\dagger$, K.B Weedop, A. Saeidpour, T.S. Brett, P. Rohani (2021).
Anomalous influenza seasonality in the United States and the emergence of novel influenza B viruses. Proceedings of the National Academy of Sciences.

\authSelf, L. Mwananyanda, W.M. MacLeod, M. Mwale, D. Thea, R.C.  Pieciak, P. Rohani, C. Gill (2020)
Implementation and adherence of routine pertussis vaccination (DTP) in a low-resource urban birth cohort. BMJ-Open.

E.M. Schultz, \authSelf, J.M. Cornelius, D.G. Reichard, K.C. Klasing, T.P. Hahn (2020). Patterns of annual and seasonal immune investment in a temporal reproductive opportunist. Proceedings of the Royal Society B.

\authSelf, K. Okamoto, H. Astete, G.M. Vasquez, E. Erhardt, 
C. Del Aguila, R. Pinedo, R. Cardenas, C. Pacheco, E. Chalco, 
H. Rodriguez-Ferruci, T.W. Scott, A.L. Lloyd, F. Gould, A.C. Morrison (2018).
Efficacy of {\em Aedes aegypti} control by indoor Ultra Low Volume (ULV)
insecticide spraying in Iquitos, Peru. PLOS Negl Trop Dis.
%Preprint available at \url{https://www.biorxiv.org/content/early/2017/12/08/231134}.

M.R. Vella$^\dagger$, \authSelf, A.L. Lloyd, F. Gould (2017).
Evaluating strategies for reversing CRISPR-Cas9 gene drives.
Scientific Reports.
%Preprint available at \url{http://www.biorxiv.org/content/early/2017/08/10/144097}.

\authSelf, M.J Ferrari, E. Erhardt, H.J. Wearing (2017).
Evidence of cryptic incidence in childhood diseases. 
Proceedings of the Royal Society B.
%Preprint available at \url{http://biorxiv.org/content/early/2016/10/04/079194}.

C. Andris, D. Lee, M.J.  Hamilton, M. Martino, \authSelf,  J.A. Selden (2015).
The Rise of Partisanship and Super-cooperators in the US House of Representatives.
PLoS ONE, 10(4), e0123507.

\authSelf, E. Erhardt, H.J. Wearing (2014). 
Conserved patterns of incomplete reporting in pre-vaccine era childhood diseases.
Proceedings of the Royal Society B 281(1794), 20140886. 

\authSelf \& H.J. Wearing (2013). Probabilistic measures of persistence
and extinction in measles (meta)populations. Ecology Letters 16(8), 985-994.

D.M. Smith, D.M. Finch, \authSelf, R. Jemison, J.F. Kelly (2009). Post-wildfire recovery of riparian vegetation during a period of water scarcity in the Southwestern USA. Fire Ecology 5(1), 38-55.
\end{rSection}


\begin{rSection}{Research Experience}

\begin{rSubsection}{Senior Research Associate}{May 2025 - present}{Odum School of Ecology, UGA}{Athens, GA}
\item Write grants and conduct independent research.
\item Focus on \textit{B. pertussis}, influenza, and SARS-CoV-2 epidemiology, ecology, and evolution.
\item Mentor undergraduate and graduate students in scientific computing and reproducible research.
\end{rSubsection}

\begin{rSubsection}{Post-doctoral Researcher}{Feb 2019 - Apr 2024}{Odum School of Ecology, UGA}{Athens, GA}
\item Data integration \& statistical analysis of historical and modern studies of pertussis, influenza, and SARS-CoV-2.
\item Lead analysis of pertussis cohort study in Lusaka, Zambia.
\item Community ecology and modeling of \textit{Salmonella} in chickens and cows.
\item Mentor undergraduate and graduate students in scientific computing and reproducible research.
\end{rSubsection}

\clearpage

\begin{rSubsection}{Post-doctoral Researcher}{Oct 2014 - July 2017}{Departments of
Entomology and Mathematics, NCSU}{Raleigh, NC}
\item Data integration \& statistical analysis of {\em Ae. aegypti} field spraying trials in Iquitos, Peru
\item Continue development of Skeeter Buster, the {\em Aedes aegypti} population dynamics simulation model
\item Mentor graduate students in Mathematics and Entomology
\end{rSubsection}

\begin{rSubsection}{Research Assistant}{Jan 2010 - Oct 2014}{Wearing Lab, UNM Biology}{Albuquerque, NM}
\item Conduct original research for publication and assist with grant writing
\item Systems administrator and data manager
\item Undergraduate training and mentorship
\end{rSubsection}

\begin{rSubsection}{Hydrology Research Technician}{Jan 2006 - Jan 2009}{Rocky Mountain Research Station, U.S. Forest Service}{Albuquerque, NM}
\item Conduct original research and prepare technical reports for U.S. Forest Service
\item Collect and managed environmental monitoring data 
\end{rSubsection}

\begin{rSubsection}{Plant Genetics Lab Technician}{Jun 2003 - Jun 2004}{Malmberg Lab, UGA Plant Biology}{Athens, GA}
\item Isolated DNA, Conducted PCR
\item Design data entry and management system
\end{rSubsection}

\begin{rSubsection}{NMR Lab Technician}{Jan 2001 - Jun 2002}{Omichinski Lab, UGA Biochemistry and Molecular Biology}{Athens, GA}
\item Administer mixed Unix workstation cluster
\item Evaluate linux hardware/software for high-performance NMR data visualization
\end{rSubsection}
\end{rSection}

%\clearpage

\begin{rSection}{Awards and Fellowships}
%\item Apr 2013 (\$500). UNM Biology Department Scholarship.
%\item Apr 2013 Graduate oral presentation, $2^{nd}$ place. UNM Biology Research Day.
%\item Apr 2010 Graduate poster presentation, $1^{st}$ place. UNM Biology Research Day.
\item Aug 2009 Program in Interdisciplinary Biological and Biomedical Sciences
(PiBBS) Fellowship. Two year fully funded tuition and graduate assistantship
stipend (2010-2012).
\end{rSection}

\begin{rSection}{Grants (Total: \$362,400)}
\item May 2022 (\$275,000). NIH R21, lead author. COVID-19 mass gatherings as natural experiments.
%\item Jun 2011 (\$2,000). UNM PiBBS Student Enrichment Grant, SFI Complex Systems Summer School.
%\item May 2010 (\$500). EEID Conference Workshop travel grant.
%\item Apr 2010 (\$500). UNM SRAC travel grant, useR 2010.
\item Mar 2010 (\$80,000). Center for Evolutionary \& Theoretical Immunology (CETI) Seed
Grant, contributing author. {\em Waning Immunity in Influenza and Whooping Cough}.
\item May 2007 (\$4,400). UNM Graduate Research and Development grant. {\em Hydrological research in the Middle Rio Grande Bosque}. 
\end{rSection}

\begin{rSection}{Contributed Talks}
\item \authSelf. \textit{Bordetella pertussis} in urban Southern Africa. International Bordetella Pertussis Lab Meeting. Feb 2024.
\item \authSelf, C.J. Gill, W.B. MacLeod, R.C. Pieciak, and Pej Rohani. Longitudinal qPCR reveals persistent asymptomatic pertussis in Zambian mothers \& infants. Annual meeting of the European Society For Paediatric Infectious Diseases. May 2021.
\item \authSelf, C.J. Gill, W.B. MacLeod, R.C. Pieciak, and Pej Rohani.  \textit{Bordetella pertussis} in urban Southern Africa: Lessons from a mother/infant cohort study. International Bordetella Pertussis Lab Meeting. Feb 2022.
\item \authSelf\ and A.L. Lloyd.  Skeeter Buster Past, Present, and Future: Challenges and Issues in Modeling Mosquito Populations.
Society of Vector Ecology. Albuquerque, NM. Sep 2015.
SAMSI Program on Mathematical and Statistical Ecology Transition Workshop. Durham, NC. May 2015.
\item \authSelf\ and H.J. Wearing.  Appropriate Measures of Persistence in Childhood Diseases.
SAMSI Program on Mathematical and Statistical Ecology Transition Workshop. Durham, NC. May 2015.
%\item \authSelf\ and H.J. Wearing.  Reporting rate variation in U.S. cities.
%UNM Biology Research Day. Albuquerque, NM. Apr 2013.
\item \authSelf.  Measles dynamics in the pre-vaccine era United States: Linking
models and data. UNM Biology Brownbag seminar series. Albuquerque, NM. Oct 2011.
\item \authSelf\ and H.J. Wearing.  Measles epidemics in pre-vaccine era United States cities: Linking models and data. Ecological Society of America conference. Austin, TX. Aug 2011.
\item \authSelf.  Spatio-temporal ecology of measles. 
UNM Biology Research Day. Albuquerque, NM. Apr 2011.
\item \authSelf. Rwave - Detecting synchrony of influenza between U.S. states.
useR 2010 Conference. Gaithersburg, MD. Jul 2010.
\end{rSection}

%\clearpage
\begin{rSection}{Contributed Posters}
\item \authSelf, C. Andris, T.S. Brett, and P. Rohani.
Signatures of focal mass gatherings: how cellular device records reveal COVID-era travel pulses. 
Ecology and Evolution of Infectious Disease (EEID) Conference. State College, PA. Jun 2023.
\item \authSelf, C.J. Gill, W.B. MacLeod, R.C. Pieciak, and P. Rohani.
Longitudinal qPCR reveals persistent asymptomatic pertussis in Zambian mothers \& infants.
Ecology and Evolution of Infectious Disease (EEID) Conference. Montpellier, France (remote). Jun 2021.
\item \authSelf, A.I. Bento, P. Rohani.  
Heterogeneous Serologic Responses to Acellular Pertussis Vaccination.
Ecology and Evolution of Infectious Disease (EEID) Conference. Princeton, NJ. Jun 2019.
\item \authSelf, E. Erhard, H.J. Wearing.  Pre-vaccine era reporting
rates of measles and whooping cough. Ecology and Evolution of Infectious Disease
(EEID) Conference. Fort Collins, CO. Jun 2014. 
\item \authSelf. Reporting rate variation of acute, immunizing diseases in
pre-vaccine U.S. cities. Ecology and Evolution of Infectious Disease
(EEID) Conference. State College, PA. May 2013. 
\item \authSelf. Stochasticity, persistence, and extinction in measles
(meta)populations. Models of Infectious Disease Agent Study
(MIDAS) meeting. Atlanta, GA. Jun 2012. 
\item \authSelf, H.J Wearing. Stochasticity, persistence, and extinction in measles
(meta)populations: Are we measuring what we think we're
measuring? Ecology \& Evolution of Infectious Disease (EEID) Conference. Ann Arbor, MI. May 2012.
\item \authSelf. Using wavelets to detect synchrony of influenza between U.S.
states. UNM Biology Research Day. Albuquerque, NM.  Apr 2010. 
\item \authSelf. Linear Modeling of the Response of Groundwater Level to River Flow
in the Middle Rio Grande Bosque, Water Year 2006. National
Groundwater Association (NGWA) Conference. Albuquerque, NM. May 2007. 
\end{rSection}

%------------------------------------------------
\begin{rSection}{Teaching}
\subhead{Instructor}
\begin{rSubsection}{Introduction to Programming in C++}{Fall 2018, Kenyon College Scientific Computing}{}{}
\nullitem
\end{rSubsection}
\begin{rSubsection}{Elements of Statistics}{Fall 2017 \& Spring 2018, Kenyon College Statistics}{}{}
\nullitem
\end{rSubsection}
\begin{rSubsection}{Intro to Experimental Biology}{Fall 2017 \& Spring 2018, Kenyon College Biology}{}{}
\nullitem
\end{rSubsection}
\begin{rSubsection}{Probability for Scientists}{Fall 2013, UNM Biology and Statistics}{}{}
\item Course designer, lead instructor
\item Mixed undergraduate/graduate course (primarily undergraduate) covering
introductory probability, statistics, and data analysis.
\end{rSubsection}
\subhead{Teaching Assistant}
\begin{rSubsection}{Biology for Non-majors}{Spring 2014, UNM Biology}{}{}
\nullitem
\end{rSubsection}
\begin{rSubsection}{Statistical Programming}{Spring 2013, UNM Statistics}{}{}
\item Mixed undergraduate/graduate course (primarily graduate). Also guest
lectured.
\end{rSubsection}
\begin{rSubsection}{Mathematical Biology}{Fall 2012, UNM Biology}{}{}
\item Mixed undergraduate/graduate course (primarily undergraduate). Also guest lectured.
\end{rSubsection}
\begin{rSubsection}{Genetics}{Spring 2009, UNM Biology}{}{}
\nullitem
\end{rSubsection}
\begin{rSubsection}{Ecology \& Evolution}{Fall 2008, UNM Biology}{}{}
\nullitem
%\item Assisted in writing course material
\end{rSubsection}
\subhead{Guest Lecturer}
\begin{rSubsection}{Ecology \& Evolution of Animal Sex}{Spring 2018, Kenyon College}{}{}
\nullitem
\end{rSubsection}
\begin{rSubsection}{Statistical Computing}{Fall 2017, Kenyon College}{}{}
\nullitem
\end{rSubsection}
\begin{rSubsection}{Theoretical Ecology}{Spring 2015, Univ. of Montana}{}{}
\nullitem
\end{rSubsection}
\subhead{Workshops and Training}
\begin{rSubsection}{Computational Skills for Scientists Training Workshop}{Aug 2016, Univ. of Montana}{}{}
\item Guest lecturer
\end{rSubsection}
\begin{rSubsection}{Industrial Math/Stat Modeling Workshop for Graduate Students}{July 2015, NCSU}{}{}
\item Guest instructor, student mentor
\end{rSubsection}
\begin{rSubsection}{Software Carpentry Workshop}{Jan 2015, NCSU}{}{}
\item Teaching assistant, guest lecturer
\end{rSubsection}
\begin{rSubsection}{UNM R Programming Group}{Fall 2010 - Spring 2013, UNM}{}{}
\item Organized and led weekly R programming group. Participants included undergraduate and graduate students and professors
\end{rSubsection}
\begin{rSubsection}{Ecology Workshop}{May 2010, Univ. of Michigan}{}{}
\item Teaching assistant
\item NSF-funded graduate training program, part of Ecology and Evolution of Infectious
Disease conference
\end{rSubsection}
\end{rSection}

\begin{rSection}{Mentoring}
\item 2021-present. Post-bacc and graduate research assistants, Rohani Lab, UGA
\item 2015 - 2020. Robert Liberatore, Software Developer
%\item 2021-2022. Elena Jauregui, UGA Ecology Post-baccalaureate research assistant.
\item 2019 - 2021. Deven Gokhale, UGA Ecology Ph.D. student
\item 2018 - 2022. Ryan Silver, Kenyon Mathematics Undegraduate student
\item 2015 - 2017. Michael Vella, NCSU Mathematics Ph.D. student
\item 2015 - 2017. Gabriel Zilnik, NCSU Entomology Masters student
\item 2011 - 2014. Undergraduate research assistants, UNM Biology
\end{rSection}

%----------------------------------------------------------------------------------------
%	WORK EXPERIENCE SECTION
%----------------------------------------------------------------------------------------
%\clearpage

%\begin{rSection}{Professional Events}
%\item Oct 2011. Rcpp R Programming Master Class. San Francisco, CA.
%\item Jun 2011. Santa Fe Institute Complex Systems Summer School. Santa Fe Institute. Santa Fe, NM. 
%\item Jun 2021. Ecology and Evolution of Infectious Disease Conference.  Montpellier, France (remote).
%\item Jun 2019. Ecology and Evolution of Infectious Disease Conference.  Princeton University. Princeton, NJ.
%\item Sep 2015. Society of Vector Ecology Conference. Albuquerque, NM.
%\item Nov 2014. American Society of Tropical Medicine and Hygiene Annual Conference.  New Orleans, LA.
%\item Jun 2014. Ecology and Evolution of Infectious Disease Conference.  Colorado State University.  Fort Collins, CO.
%\item May 2013. Ecology and Evolution of Infectious Disease Conference.  Pennsylvania State University.  State College, PA.
%\item May 2012. Ecology and Evolution of Infectious Disease Conference and Workshop.  University of Michigan.  Ann Arbor, MI.
%\item Aug 2011. Ecological Society of America Conference. Austin, TX.
%\item Jul 2010. useR 2010 Conference. Gaithersburg, MD. 
%\item May 2010. Ecology and Evolution of Infectious Disease Conference and Workshop.  Cornell University.  Ithica, NY.
%\item Aug 2009. Ecological Society of America Conference. Albuquerque, NM.
%\item May 2007. National Groundwater Association Conference. Albuquerque, NM. 
%\end{rSection}

\begin{rSection}{Professional Service}
\item Reviewer, Ecology Letters.
\item Reviewer, PLOS Neglected Tropical Diseases
\item Reviewer, Scientific Reports
\item Reviewer, Journal of Pediatric Infectious Diseases
\item Reviewer, Journal of the Royal Society Interface
\item Reviewer, BMC Infectious Diseases
\item Reviewer, PLOS Computational Biology
\item Reviewer, Transactions of the Royal Society of Tropical Medicine and Hygiene.
\item Reviewer, Theoretical Ecology.
\item Grant review, Digital Technology Development Award. Wellcome Trust.
\item Grant reader, Graduate Research Allocations Committee (GRAC). UNM Biology.
\end{rSection}

\begin{rSection}{Software Development}
\item Fall 2014 - Winter 2019. Develop and maintain Skeeter Buster: a stochastic,
spatially-explicit, agent-based C++ simulation model of {\em Aedes aegypti}
population dynamics.
\item Spring 2013. Wrote code, documentation, and tests according to
specifications of Drs. J. M. Rowland and C. Qualls for \texttt{discrimArts}: R package for probability
distribution estimation.
\item 2010 - 2012. Contributor to \texttt{Rcpp}: R package for C++ development.
\item 2009 - 2012. Maintainer of \texttt{Rwave}: R package for continuous wavelet transforms.
\item 2010 - 2011. Contributor to \texttt{xts} and \texttt{zoo}: R packages for time series handling and analysis.
\end{rSection}
\end{document}
