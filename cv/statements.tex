My research explores the ecological determinants of real-world population
dynamics by combining empirical data, numerical simulations, and statistical
models.  I am especially interested in stochastic extinction and persistence,
partially observed Markov processes, and population responses to perturbation.
At present, my study systems include childhood diseases in the pre-vaccine era
U.S., and mosquito population dynamics in Iquitos, Peru.

My core areas of expertise include: stochastic processes, numerical
programming, and mechanistic simulation models;
data visualization; time series
spectral analysis; and generalized linear modeling.  I favor open source
computing tools: the R and C++ programming languages; PostgreSQL and SQLite
database systems; and the geospatial tools PostGIS and QGIS.  I employ
reproducible research tools and best practices, including revision control with
git.

I enjoy teaching and mentoring students from diverse backgrounds and
experience levels, with a focus on quantitative skills, real-world examples, and
student-led inquiry.  I believe that rigorous quantitative training is vital
to the next generation of data-driven scientists.
