%%%%%%%%%%%%%%%%%%%%%%%%%%%%%%%%%%%%%%%%%
% Medium Length Professional CV
% LaTeX Template
% Version 2.0 (8/5/13)
%
% This template has been downloaded from:
% http://www.LaTeXTemplates.com
%
% Original author:
% Trey Hunner (http://www.treyhunner.com/)
%
% Important note:
% This template requires the resume.cls file to be in the same directory as the
% .tex file. The resume.cls file provides the resume style used for structuring the
% document.
%
%%%%%%%%%%%%%%%%%%%%%%%%%%%%%%%%%%%%%%%%%

%----------------------------------------------------------------------------------------
%	PACKAGES AND OTHER DOCUMENT CONFIGURATIONS
%----------------------------------------------------------------------------------------

\documentclass{resume} % Use the custom resume.cls style

\usepackage[left=0.75in,top=0.6in,right=0.75in,bottom=0.6in]{geometry} % Document margins
%% footer
\usepackage{fancyhdr}
% clear old, activate new footer
\fancyhf{} 
\pagestyle{fancy}

\name{Christian E. Gunning} % Your name
\address{(706)~224-7627 \\ http://www.x14n.org \\ icos.atropa@gmail.com} % Your phone number and email
\address{307 A Princeton SE \\ Albuquerque, NM 87106} % Your address
%Updated \todoy

\begin{document}

%% footer
\rfoot{C. E. Gunning, Page \thepage}
%----------------------------------------------------------------------------------------
%	EDUCATION SECTION
%----------------------------------------------------------------------------------------


\begin{rSection}{Background}
%\begin{itemize}
\item Research employs a dynamical systems interpretation of
population and community ecology, with an emphasis on probabilistic
analysis of empirical datasets and discrete stochastic dynamical models.
\item Current focus on metapopulation dynamics of measles and whooping
cough, including the effect of population size on stochastic
extinction as a metapopulation measure of disease persistence.
\item Strong background in probability theory, machine learning
methods, linear modeling, and bootstrapping.  Fluent in the R programming language, with extensive experience in SQL, C++, and multidimensional data visualization.
\item Skilled mentor and teacher of
probability, statistics, and scientific programming to students from diverse
backgrounds. 
\end{rSection}

\begin{rSection}{Education}
{\bf University of New Mexico, Albuquerque} \hfill {\em Summer 2014} \\ 
Ph.D. with Distinction in Biology (Disease Ecology with concentration in
Integrative Biology) \\
Advisor: Dr. Helen J. Wearing\\
Committee: Drs. Jim Brown, Melanie Moses, and Erik Erhardt\\
Title: Population and Metapopulation Ecology of
Childhood Diseases in the pre-vaccine era United States

{\bf University of New Mexico, Albuquerque} \hfill {\em Fall 2009} \\ 
Masters of Water Resources (Riparian Hydroecology) \\
Advisors: Drs. Bruce Thomson and Roy Jemison \\
Title: Estimating Phreatophyte Evapotranspiration from Diel Groundwater Fluctuations in the Middle Rio Grande Bosque

{\bf University of Georgia, Athens} \hfill {\em Fall 2001} \\ 
Bachelor of Science, Biochemistry and Molecular Biology \\
Advisor: Dr. James Omichinski
\end{rSection}

\begin{rSection}{Grants and Awards}
\item Apr 2013 (\$500). UNM Biology Department Scholarship.
\item Apr 2013. Graduate oral presentation, $2^{nd}$ place. UNM Biology Research Day.
\item Jun 2011 (\$2,000). UNM PiBBS Student Enrichment Opportunities Grant to
attend SFI Complex Systems Summer School.
\item May 2010 (\$500). EEID Conference Workshop travel grant.
\item Apr 2010 (\$500). UNM SRAC travel grant to attend useR2010.
\item Apr 2010.  Graduate poster presentation, $1^{st}$ place. UNM Biology Research Day.
\item Mar 2010 (\$80,000). Center for Evolutionary \& Theoretical Immunology (CETI) Seed
Grant, Waning Immunity in Influenza and Whooping Cough,
Contributing author.
\item Aug 2009 ($\approx$\$50,000). Program in Interdisciplinary Biological and Biomedical
Sciences (PIBBS) 2 year fellowship.
\item May 2007 (\$4,400). UNM Graduate Research and Development grant, Hydrological
research in the Middle Rio Grande Bosque. 
\end{rSection}

\begin{rSection}{Publications}
{\bf C.E. Gunning}, E. Erhardt, H.J. Wearing.  Pre-vaccine era reporting rates
of childhood diseases: a case study of observation process variability.  (In
Review, Proc. Roy. Soc. Lond. B.)

C. Andris, D. Lee, {\bf C.E. Gunning}, M. Martino, M.J.  Hamilton, J.A. Selden.
The Rise of Partisanship in the U.S. House of Representatives.  (In Review, PLoS
ONE)

{\bf C.E. Gunning} \& H.J. Wearing. Probabilistic measures of persistence
and extinction in measles (meta)populations.  Ecology Letters 16(8), 985-994.

D.M. Smith, D.M. Finch, {\bf C.E. Gunning}, R. Jemison, J.F. Kelly. 2009. Post-wildfire recovery of riparian vegetation during a period of water scarcity in the Southwestern USA. Fire ecology 5(1), 38-55.
\end{rSection}


\begin{rSection}{Contributed Talks}
\item {\bf C.E. Gunning}, H.J Wearing.  Reporting rate variation in U.S. cities.
UNM Biology Research Day. Albuquerque, NM. Apr 2013.
\item {\bf C.E. Gunning}.  Measles dynamics in the pre-vaccine era United States: Linking
models and data. UNM Biology Brownbag seminar series. Albuquerque, NM. Oct 2011.
\item {\bf C.E. Gunning}, H.J Wearing.  Measles epidemics in pre-vaccine era United States cities: Linking models and data. Ecological Society of America conference. Austin, TX. Aug 2011.
\item {\bf C.E. Gunning}.  Spatio-temporal ecology of measles. 
UNM Biology Research Day. Albuquerque, NM. Apr 2011.
\item {\bf C.E. Gunning}. Rwave - Detecting synchrony of influenza between U.S. states.
useR 2010 Conference. Gaithersburg, MD. Jul 2010.
\end{rSection}

\begin{rSection}{Contributed Posters}
\item {\bf C.E. Gunning}, E. Erhard, H.J. Wearing.  Pre-vaccine era reporting
rates of measles and whooping cough. Ecology and Evolution of Infectious Disease
(EEID) Conference. Fort Collins, CO. Jun 2014. 
\item {\bf C.E. Gunning}. Reporting rate variation of acute, immunizing diseases in
pre-vaccine U.S. cities. Ecology and Evolution of Infectious Disease
(EEID) Conference. State College, PA. May 2013. 
\item {\bf C.E. Gunning}. Stochasticity, persistence, and extinction in measles
(meta)populations. Models of Infectious Disease Agent Study
(MIDAS) meeting. Atlanta, GA. Jun 2012. 
\item {\bf C.E. Gunning}, H.J Wearing. Stochasticity, persistence, and extinction in measles
(meta)populations: Are we measuring what we think we're
measuring? Ecology \& Evolution of Infectious Disease (EEID) Conference. Ann Arbor, MI. May 2012.
\item {\bf C.E. Gunning}. Using wavelets to detect synchrony of influenza between U.S.
states. UNM Biology Research Day. Albuquerque, NM.  Apr 2010. 
\item {\bf C.E. Gunning}. Linear Modeling of the Response of Groundwater Level to River Flow
in the Middle Rio Grande Bosque, Water Year 2006. National
Groundwater Association (NGWA) Conference. Albuquerque, NM. May 2007. 
\end{rSection}


%------------------------------------------------
\begin{rSection}{Teaching Experience}
\begin{rSubsection}{Probability for Scientists}{Fall 2013, UNM Biology}{}{}
\item Course designer and lead instructor
\item Mixed undergraduate/graduate course (primarily undergraduate)
\item Hands-on course covering introductory probability, statistics, and data analysis
\end{rSubsection}
\begin{rSubsection}{Statistical Programming}{Spring 2013, UNM Statistics}{}{}
\item Teaching assistant
\item Mixed undergraduate/graduate course (primarily graduate)
\item Assisted Dr. Erik Erhardt with course lectures
\end{rSubsection}
\begin{rSubsection}{UNM R Programming Group}{Fall 2010 - Spring 2013, UNM}{}{}
\item Organizer and facilitator of weekly R programming group
\item Participants included undergraduate and graduate students and professors
\end{rSubsection}
\begin{rSubsection}{Ecology Workshop}{May 2010, University of Michigan}{}{}
\item Assisted workshop participants with use of R programming language
\item Workshop was part of the annual Ecology and Evolution of Infectious
Disease conference
\end{rSubsection}
\begin{rSubsection}{Undergraduate Biology Courses (various)}{2008, 2009, 2014, UNM Biology}{}{}
\item Teaching assistant for Ecology \& Evolution, Genetics, and Biology for
Non-majors
\end{rSubsection}
\end{rSection}

%----------------------------------------------------------------------------------------
%	WORK EXPERIENCE SECTION
%----------------------------------------------------------------------------------------

\begin{rSection}{Research Experience}
\begin{rSubsection}{Research Assistant}{Jan 2010 - Present}{Wearing Lab, UNM Biology}{Albuquerque, NM}
\item Conducted original research for publication
\item Assisted with grant writing
\item Systems administrator and data manager
\item Mentor and manager of lab undergraduates
\end{rSubsection}


\begin{rSubsection}{Hydrology Research Technician}{Jan 2006 - Jan 2009}{Rocky
Mountain Research Station, U.S. Forest Service}{Albuquerque, NM}
\item Conducted original research for U.S. Forest Service
\item Prepared technical reports
\item Collected and managed environmental monitoring data 
\end{rSubsection}

\begin{rSubsection}{Plant Genetics Lab Technician}{Jun 2003 - Jun 2004}{Malmberg Lab, UGA Plant Biology}{Athens, GA}
\item Isolated DNA, Conducted PCR
\item Designed data entry and management system
\end{rSubsection}

\begin{rSubsection}{NMR Lab Technician}{Jan 2001 - Jun 2002}{Omichinski Lab, UGA
Biochemistry and Molecular Biology}{Athens, GA}
\item Administered mixed Unix workstation cluster
\item Evaluated linux hardware/software for high-performance NMR data
visualization
\end{rSubsection}
\end{rSection}



\clearpage

\begin{rSection}{Conferences and Professional Events}
\item Jun 2014. Ecology and Evolution of Infectious Disease Conference.
Colorado State University.  Fort Collins, CO.
\item May 2013. Ecology and Evolution of Infectious Disease Conference.
Pennsylvania State University.  State College, PA.
\item May 2012. Ecology and Evolution of Infectious Disease Conference
and Workshop.  University of Michigan.  Ann Arbor, MI.
\item Oct 2011. Rcpp R Programming Master Class. San Francisco, CA.
\item Aug 2011. Ecological Society of America Conference. Austin, TX.
\item Jun 2011. Santa Fe Institute Complex Systems Summer School. Santa Fe Institute. Santa Fe, NM. 
\item Jul 2010.  useR 2010 Conference. Gaithersburg, MD. 
\item May 2010. Ecology and Evolution of Infectious Disease Conference
and Workshop.  Cornell University.  Ithica, NY.
\item Aug 2009. Ecological Society of America Conference. Albuquerque, NM.
\item May 2007. National Groundwater Association Conference. Albuquerque, NM. 
\end{rSection}

\begin{rSection}{Professional Service}
\item Feb 2014.  Reviewer, Theoretical Ecology.
\item May 2011 - May 2012. Mentored undergraduate Nathan Cournoyer. UNM Biology.
\item Mar 2009. Grant reader, Graduate Research Allocations Committee (GRAC). UNM Biology.
\end{rSection}

\begin{rSection}{Software Development}
\item Spring 2013. Wrote code, documentation, and unit tests for \texttt{discrimArts}, an R package for probability
distribution estimation, according to the specifications of Drs. J. M. Rowland and C. Qualls.
\item Oct 2010 - Mar 2012. Contributor to \texttt{Rcpp},  an R package for C++ development.
\item 2009-2012. Maintainer of \texttt{Rwave}, an R package for continuous
wavelet transforms.
\item 2010-2011. Submitted bug reports for \texttt{xts} and \texttt{zoo}, R
packages for timeseries handling and analysis.
\end{rSection}
%----------------------------------------------------------------------------------------
%	EXAMPLE SECTION
%----------------------------------------------------------------------------------------

%\begin{rSection}{Section Name}

%Section content\ldots

%\end{rSection}

%----------------------------------------------------------------------------------------

\end{document}
