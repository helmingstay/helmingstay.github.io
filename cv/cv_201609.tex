%%%%%%%%%%%%%%%%%%%%%%%%%%%%%%%%%%%%%%%%%
% Medium Length Professional CV
% LaTeX Template
% Version 2.0 (8/5/13)
%
% This template has been downloaded from:
% http://www.LaTeXTemplates.com
%
% Original author:
% Trey Hunner (http://www.treyhunner.com/)
%
% Important note:
% This template requires the resume.cls file to be in the same directory as the
% .tex file. The resume.cls file provides the resume style used for structuring the
% document.
%
%%%%%%%%%%%%%%%%%%%%%%%%%%%%%%%%%%%%%%%%%

%----------------------------------------------------------------------------------------
%	PACKAGES AND OTHER DOCUMENT CONFIGURATIONS
%----------------------------------------------------------------------------------------

\documentclass{resume} % Use the custom resume.cls style

\usepackage[left=0.75in,top=0.6in,right=0.75in,bottom=0.6in]{geometry} % Document margins
%% footer
\usepackage{fancyhdr}
\pagestyle{fancy}
\fancyhf{} 
\renewcommand{\headrulewidth}{0pt}
\rfoot{C.E. Gunning, Page \thepage}
% clear old, activate new footer
%\pagestyle{fancy}

\name{Christian E. Gunning} % Your name
\address{(706)~224-7627 \\ www.x14n.org \\ research@x14n.org} % Your phone number and email
%\address{307 A Princeton SE \\ Albuquerque, NM 87106} % Your address
%Updated \todoy

\begin{document}

%----------------------------------------------------------------------------------------
%	EDUCATION SECTION
%----------------------------------------------------------------------------------------


\begin{rSection}{Executive Summary}
%\begin{itemize}
\item 
My research examines the spatiotemporal dynamics of population ecology, 
with a focus on systems and processes that are discrete, stochastic,
and non-linear. As a quantitative ecologist, I draw 
inspiration and methodology from a range of disciplines, including 
chemistry, physics, and mathematics.
I use empirical data, numerical simulations, and modern statistical methods to
identify key determinants of population dynamics.  I am particularly interested
in stochastic extinction and persistence, population response to
perturbation, and demographic founder events such
as colonization and development of resistance. At present, my study systems 
include childhood diseases in the pre-vaccine era U.S.\, and mosquito population 
dynamics in Iquitos, Peru.

\item My research is facilitated by cutting-edge quantitative tools:
the R and C++ programming languages for data visualization,
numerical simulation, and statistical modeling, as well as SQL for 
advanced data management. I use reproducible research best practices 
to maximize work efficiency and quality.

\item I have extensive experience teaching and mentoring students from diverse
backgrounds, and believe that student training is critical to developing the
next generation of data-driven scientists.  In particular, I aim to increase
student access to probability and statistics, scientific programming,
and reproducible research methods.

\end{rSection}

\begin{rSection}{Education}
{\bf University of New Mexico, Albuquerque} \hfill {\em Summer 2014} \\ 
Ph.D. with Distinction in Biology (Disease Ecology with concentration in
Integrative Biology) \\
Committee: Drs. Helen J. Wearing (advisor), Jim Brown, Melanie Moses, and Erik Erhardt\\
Title: Population and metapopulation ecology of childhood diseases in the pre-vaccine era United States

{\bf University of New Mexico, Albuquerque} \hfill {\em Fall 2009} \\ 
Masters of Water Resources (Riparian Hydroecology) \\
Advisors: Drs. Bruce Thomson and Roy Jemison \\
Title: Estimating pHreatophyte evapotranspiration from diel groundwater fluctuations in the Middle Rio Grande Bosque

{\bf University of Georgia, Athens} \hfill {\em Fall 2001} \\ 
Bachelor of Science, Biochemistry and Molecular Biology \\
Advisor: Dr. James Omichinski
\end{rSection}


\begin{rSection}{Peer-Reviewed Publications}

C. Andris, D. Lee, M.J.  Hamilton, M. Martino, {\bf C.E. Gunning},  J.A. Selden (2015).
The Rise of Partisanship and Super-cooperators in the US House of Representatives.
PLoS ONE, 10(4), e0123507.

{\bf C.E. Gunning}, E. Erhardt, H.J. Wearing (2014). 
Conserved patterns of incomplete reporting in pre-vaccine era childhood diseases.
Proceedings of the Royal Society B 281(1794), 20140886. 

{\bf C.E. Gunning} \& H.J. Wearing. Probabilistic measures of persistence
and extinction in measles (meta)populations (2013).  Ecology Letters 16(8), 985-994.

D.M. Smith, D.M. Finch, {\bf C.E. Gunning}, R. Jemison, J.F. Kelly (2009). Post-wildfire recovery of riparian vegetation during a period of water scarcity in the Southwestern USA. Fire Ecology 5(1), 38-55.
\end{rSection}

\begin{rSection}{Pending Publications}

M.R. Vella, {\bf C.E. Gunning}, A.L. Lloyd, F. Gould.
Evaluating strategies for reversing CRISPR-Cas9 gene drives.
Nature Biotechnology, In Review.

{\bf C.E. Gunning}, K. Okamoto, H. Astete, G.M. Vasquez, E. Erhardt, 
C. Del Aguila, R. Pinedo, R. Cardenas, C. Pacheco, E. Chalco, 
H. Rodriguez-Ferruci, T.W. Scott, A.L. Lloyd, F. Gould, A.C. Morrison.
Efficacy of {\em Aedes aegypti} control by indoor Ultra Low Volume (ULV) spraying 
in Iquitos, Peru. In Prep.

{\bf C.E. Gunning}, M.J Ferrari, E. Erhardt, H.J. Wearing.
Evidence of cryptic incidence in childhood diseases. In Prep.

\end{rSection}

\begin{rSection}{Research Experience}

\begin{rSubsection}{Post-doctoral Researcher}{Oct 2014 - Present}{Departments of
Entomology and Mathematics, NCSU}{Raleigh, NC}
\item Statistical analysis of {\em Aedes aegypti} field spraying trials in Iquitos, Peru
\item Continue development of Skeeter Buster of {\em Aedes aegypti} population dynamics simulation model
\item Mentor graduate students
\end{rSubsection}

\begin{rSubsection}{Research Assistant}{Jan 2010 - Oct 2014}{Wearing Lab, UNM Biology}{Albuquerque, NM}
\item Conduct original research for publication and assist with grant writing
\item Systems administrator and data manager
\item Undergraduate training and mentorship
\end{rSubsection}

\begin{rSubsection}{Hydrology Research Technician}{Jan 2006 - Jan 2009}{Rocky
Mountain Research Station, U.S. Forest Service}{Albuquerque, NM}
\item Conduct original research and prepare technical reports for U.S. Forest Service
\item Collect and managed environmental monitoring data 
\end{rSubsection}

\begin{rSubsection}{Plant Genetics Lab Technician}{Jun 2003 - Jun 2004}{Malmberg Lab, UGA Plant Biology}{Athens, GA}
\item Isolated DNA, Conducted PCR
\item Design data entry and management system
\end{rSubsection}

\begin{rSubsection}{NMR Lab Technician}{Jan 2001 - Jun 2002}{Omichinski Lab, UGA
Biochemistry and Molecular Biology}{Athens, GA}
\item Administer mixed Unix workstation cluster
\item Evaluate linux hardware/software for high-performance NMR data visualization
\end{rSubsection}
\end{rSection}

\begin{rSection}{Grants}
\item Jun 2011 (\$2,000). UNM PiBBS Student Enrichment Opportunities Grant to
\item May 2010 (\$500). EEID Conference Workshop travel grant.
\item Mar 2010 (\$80,000). Center for Evolutionary \& Theoretical Immunology (CETI) Seed
Grant, Waning Immunity in Influenza and Whooping Cough,
Contributing author.
\item Aug 2009 ($\approx$\$50,000). Program in Interdisciplinary Biological and Biomedical
Sciences (PIBBS) 2 year fellowship.
\item May 2007 (\$4,400). UNM Graduate Research and Development grant, Hydrological
research in the Middle Rio Grande Bosque. 
\end{rSection}

\begin{rSection}{Awards}
\item Apr 2013 (\$500). UNM Biology Department Scholarship.
\item Apr 2013. Graduate oral presentation, $2^{nd}$ place. UNM Biology Research Day.
attend SFI Complex Systems Summer School.
\item Apr 2010 (\$500). UNM SRAC travel grant to attend useR2010.
\item Apr 2010.  Graduate poster presentation, $1^{st}$ place. UNM Biology Research Day.
\end{rSection}

\begin{rSection}{Contributed Talks}
\item {\bf C.E. Gunning}, A.L. Lloyd.  Skeeter Buster Past, Present, and Future: Challenges and Issues in Modeling Mosquito Populations.
Society of Vector Ecology. Albuquerque, NM. Sep 2015.
SAMSI Program on Mathematical and Statistical Ecology Transition Workshop. Durham, NC. May 2015.
\item {\bf C.E. Gunning}, H.J. Wearing.  Appropriate Measures of Persistence in Childhood Diseases.
SAMSI Program on Mathematical and Statistical Ecology Transition Workshop. Durham, NC. May 2015.
\item {\bf C.E. Gunning}, H.J. Wearing.  Reporting rate variation in U.S. cities.
UNM Biology Research Day. Albuquerque, NM. Apr 2013.
\item {\bf C.E. Gunning}.  Measles dynamics in the pre-vaccine era United States: Linking
models and data. UNM Biology Brownbag seminar series. Albuquerque, NM. Oct 2011.
\item {\bf C.E. Gunning}, H.J. Wearing.  Measles epidemics in pre-vaccine era United States cities: Linking models and data. Ecological Society of America conference. Austin, TX. Aug 2011.
\item {\bf C.E. Gunning}.  Spatio-temporal ecology of measles. 
UNM Biology Research Day. Albuquerque, NM. Apr 2011.
\item {\bf C.E. Gunning}. Rwave - Detecting synchrony of influenza between U.S. states.
useR 2010 Conference. Gaithersburg, MD. Jul 2010.
\end{rSection}

\begin{rSection}{Contributed Posters}
\item {\bf C.E. Gunning}, E. Erhard, H.J. Wearing.  Pre-vaccine era reporting
rates of measles and whooping cough. Ecology and Evolution of Infectious Disease
(EEID) Conference. Fort Collins, CO. Jun 2014. 
\item {\bf C.E. Gunning}. Reporting rate variation of acute, immunizing diseases in
pre-vaccine U.S. cities. Ecology and Evolution of Infectious Disease
(EEID) Conference. State College, PA. May 2013. 
\item {\bf C.E. Gunning}. Stochasticity, persistence, and extinction in measles
(meta)populations. Models of Infectious Disease Agent Study
(MIDAS) meeting. Atlanta, GA. Jun 2012. 
\item {\bf C.E. Gunning}, H.J Wearing. Stochasticity, persistence, and extinction in measles
(meta)populations: Are we measuring what we think we're
measuring? Ecology \& Evolution of Infectious Disease (EEID) Conference. Ann Arbor, MI. May 2012.
\item {\bf C.E. Gunning}. Using wavelets to detect synchrony of influenza between U.S.
states. UNM Biology Research Day. Albuquerque, NM.  Apr 2010. 
\item {\bf C.E. Gunning}. Linear Modeling of the Response of Groundwater Level to River Flow
in the Middle Rio Grande Bosque, Water Year 2006. National
Groundwater Association (NGWA) Conference. Albuquerque, NM. May 2007. 
\end{rSection}

\clearpage
%------------------------------------------------
\begin{rSection}{Teaching}
\subhead{Instructor}
\begin{rSubsection}{Probability for Scientists}{Fall 2013, UNM Biology}{}{}
\item Course designer, lead instructor
\item Mixed undergraduate/graduate course (primarily undergraduate)
\item Hands-on course covering introductory probability, statistics, and data analysis
\end{rSubsection}
\subhead{Teaching Assistant}
\begin{rSubsection}{Biology for Non-majors}{Spring 2014}{}{}
\nullitem
\end{rSubsection}
\begin{rSubsection}{Statistical Programming}{Spring 2013, UNM Statistics}{}{}
\item Mixed undergraduate/graduate course (primarily graduate)
\item Also guest lectured
\end{rSubsection}
\begin{rSubsection}{Genetics}{Spring 2009, UNM Biology}{}{}
\nullitem
\end{rSubsection}
\begin{rSubsection}{Ecology \& Evolution}{Fall 2008, UNM Biology}{}{}
\item Assisted in writing course material
\end{rSubsection}
\subhead{Guest Lecturer}
\begin{rSubsection}{Theoretical Ecology}{Spring 2015, University of Montana}{}{}
\item Also assisted students with R
\end{rSubsection}
\subhead{Workshops and Training}
\begin{rSubsection}{Computational Skills for Scientists Training Workshop}{Aug 2016, Univ. of Montana}{}{}
\item Guest lecturer
\end{rSubsection}
\begin{rSubsection}{Industrial Math/Stat Modeling Workshop for Graduate Students}{July 2015, NCSU}{}{}
\item Guest instructor, student mentor
\end{rSubsection}
\begin{rSubsection}{Software Carpentry Workshop}{Jan 2015, NCSU}{}{}
\item Teaching assistant, guest lecturer
\end{rSubsection}
\begin{rSubsection}{UNM R Programming Group}{Fall 2010 - Spring 2013, UNM}{}{}
\item Organized and led weekly R programming group
\item Participants included undergraduate and graduate students and professors
\end{rSubsection}
\begin{rSubsection}{Ecology Workshop}{May 2010, Univ. of Michigan}{}{}
\item Teaching assistant
\item NSF-funded graduate training program, part of Ecology and Evolution of Infectious
Disease conference
\end{rSubsection}
\end{rSection}

\begin{rSection}{Mentoring}
\item Spring 2015 - present. Robert Liberatore, Math Education Software Developer
\item Fall 2015 - present. Michael Vella, NCSU Mathematics Ph.D. Student
\item Fall 2015 - Fall 2016. Gabriel Zilnik, NCSU Entomology Masters Student
\item Spring 2011 - Spring 2012. Nathan Cournoyer, UNM Biology Undergraduate Student
\end{rSection}

%----------------------------------------------------------------------------------------
%	WORK EXPERIENCE SECTION
%----------------------------------------------------------------------------------------



\begin{rSection}{Conferences and Professional Events}
\item Sep 2015. Society of Vector Ecology Conference.  Albuquerque, NM.
\item Nov 2014. American Society of Tropical Medicine and Hygiene Annual
Conference.  New Orleans, LA.
\item Jun 2014. Ecology and Evolution of Infectious Disease Conference.
Colorado State University.  Fort Collins, CO.
\item May 2013. Ecology and Evolution of Infectious Disease Conference.
Pennsylvania State University.  State College, PA.
\item May 2012. Ecology and Evolution of Infectious Disease Conference
and Workshop.  University of Michigan.  Ann Arbor, MI.
\item Oct 2011. Rcpp R Programming Master Class. San Francisco, CA.
\item Aug 2011. Ecological Society of America Conference. Austin, TX.
\item Jun 2011. Santa Fe Institute Complex Systems Summer School. Santa Fe Institute. Santa Fe, NM. 
\item Jul 2010.  useR 2010 Conference. Gaithersburg, MD. 
\item May 2010. Ecology and Evolution of Infectious Disease Conference
and Workshop.  Cornell University.  Ithica, NY.
\item Aug 2009. Ecological Society of America Conference. Albuquerque, NM.
\item May 2007. National Groundwater Association Conference. Albuquerque, NM. 
\end{rSection}

\begin{rSection}{Professional Service}
\item 2015. Reviewer, Ecology Letters.
\item 2014. Reviewer, Theoretical Ecology.
\item Mar 2009. Grant reader, Graduate Research Allocations Committee (GRAC). UNM Biology.
\end{rSection}

\begin{rSection}{Software Development}
\item Fall 2014 - Present. Develop and maintain Skeeter Buster: a stochastic,
spatially-explicit, agent-based C++ simulation model of {\em Aedes aegypti}
population dynamics.
\item Spring 2013. Wrote code, documentation, and tests according to
specifications of Drs. J. M. Rowland and C. Qualls for \texttt{discrimArts}: R package for probability
distribution estimation.
\item Oct 2010 - Mar 2012. Contributor to \texttt{Rcpp}: R package for C++ development.
\item 2009-2012. Maintainer of \texttt{Rwave}: R package for continuous wavelet transforms.
\item 2010-2011. Contributor to \texttt{xts} and \texttt{zoo}: R packages for time series handling and analysis.
\end{rSection}
%----------------------------------------------------------------------------------------
%	EXAMPLE SECTION
%----------------------------------------------------------------------------------------

%\begin{rSection}{Section Name}

%Section content\ldots

%\end{rSection}

%----------------------------------------------------------------------------------------

\end{document}
